
% Experement data + calcs goes here %

\section{Ход работы}

\subsection{Определение разрешенных направлений поляроида}

Определил разрешенное направление поляроида. Свет поляризуется по "ручкам" поляроида.
По лимбу $ x = (2\pm) ^\circ$. \\

Определил так же разрешенное направление второго поляроида, заменив им черное заркало.
Свет так же поляризуется по "ручкам" поляроида. По лимбу $ x = (83\pm2) ^\circ $.

\subsection{Определение показателя преломления эбонита}

Поставил на скамью вместо черного зеркала эбонитовую пластину. Определил по Лимбу угол
Брюстера $ \Theta = (56\pm4) ^\circ $. \\

Тогда $ n = \tan(\Theta) \approx 1,6\pm0,15 $.

\subsection{Исследование стопы Столетова}

Исследовал характер поляризации в преломленном и отраженном от стопы лучах. Для этого
поместил вместо эбонитовой пластины стопу стеклянных пластин под углом Брюстера. \\

Рассмотрел лучи через поляроид. Таким образом определил в них ориентацию вектора
$ \textbf{E} $. Преломленные лучи поляризованы горизонтально. Отраженные - вертикально.

\subsection{Двоякопреломляющие пластины}

Определю главные направления двоякопреломляющих пластин. \\

Для этого поместил пластину между двумя скрещенными поляроидами. После этого, вращая
пластину вокруг направления луча, определю, какие напраления пластины совпадают с
разрешенными направлениями поляроидов. \\

Минимумы и максимумы интенсивности чередуются с периодом в $ 45^\circ $. Главные плоскости
пластин совпадают с разрешенными направлениями поляроидов при достижении максимальной
интенсивности.

\subsection{Пластины $ \lambda/2 \texttt{и} \lambda/4 $}

Для выделения пластин $ \lambda/2 \texttt{и} \lambda/4 $, добавил к схеме зеленый
светофильтр и установил разрешенное направление первого поляроида горизонтально, а
главные направления исследуемой пластины под $ 45^\circ $. \\

Если свет, прошедший через второй поляроид имеет круговую поляризацию, то это пластина
$ \lambda/4 $. Если поляризация линейная с переходом в другой квадрант, то это
$ \lambda/2 $. \\

\subsection{Быстрая и медленная оси $ \lambda/4 $}

Поставил между скрещенными поляроидами пластинку чувствительного оттенка, имеющую вид
стрелки, и убедился, что эта пластинка не меняет поляризацию зелёного света. Уберал зелёный
фильтр и убедился, что стрелка имеет пурпурный цвет. Это объясняется тем, что зелёная
компонента линейно поляризованного света при прохождении пластинки не меняет поляризации
и задерживается вторым поляроидом. \\

Добавил к схеме пластинку $ \lambda/4 $, главные направления которой совпадают с главными
направлениями пластины $ \lambda $ и ориентированы под углом $ 45^\circ $ к разрешённым
направлениям скрещенных поляроидов. При повороте рейтера со стрелкой на $ 180^\circ $
вокруг вертикальной оси цвет стрелки меняется от зелёно-голубого до оранжево-жёлтого.
В случае зелено-голубого цвета - "быстрая" ось. Иначе --- "медленная".

\subsection{Интерференция поляризованных лучей}

Исследовал интерференцию поляризованных лучей. Для этого расположил между скрещенными
поляроидами мозаичную пластинку. \\

Вращая пластинку, пронаблюдал за изменениями в отдельном квадрате. Изменяется интенсивность. \\

Не трогая пластинки, провращал второй поляроид. Отличие в том, что теперь изменяется цвет.

\subsection{Эллиптически поляризованная волна}

Нарисовал эллипс поляризации для вектора напряжённости из пластинки $ \lambda/4 $ и указал,
какая из осей соответствует большей скорости. Это ось $ x $. \\

Рядом нарисовал две вышедших из пластинки синусоиды: $ x(t) $ и $ y(t) $ со сдвигом фаз в
четверть периода. \\

Определил направление вращения электрического вектора в эллиптически поляризованной волне.

\begin{center}
\begin{minipage}{7cm}
    \pic{0.9\linewidth}{drawings1.jpg}{Эллипс поляризации}
\end{minipage}
\begin{minipage}{7cm}
    \pic{0.9\linewidth}{drawings2.jpg}{Графики синусоид}
\end{minipage}
\end{center}

\section{Вывод}

В ходе данной работы были проверены различные свойства поляризации света. Также были вычислены
оптические характеристики материалов, связанные с поляризацией
