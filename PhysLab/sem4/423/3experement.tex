
% Experement data + calcs goes here %

\section{Ход работы}

Средняя длинна волны пропускания компенсатора: $ \lambda\ = 670 $ нм; \\
Длинна кюветы: $ L = 25 $ см.

\subsection{Калибровка компенсатора}

Откалибровал компенсатор с использованием светофильтра 6200-7200. Данные представлены
в Таблице 1.

\begin{table}[h!]
    \begin{center}
    Таблица 1. \\
    \begin{tabular}{|l|l|l|l|}
    \hline
    m   & z, мм & m  & z, мм \\ \hline
    -10 & -0,97 & 1  & 2,81  \\ \hline
    -9  & -0,66 & 2  & 3,19  \\ \hline
    -8  & -0,35 & 3  & 3,53  \\ \hline
    -7  & 0,02  & 4  & 3,85  \\ \hline
    -6  & 0,36  & 5  & 4,22  \\ \hline
    -5  & 0,67  & 6  & 4,58  \\ \hline
    -4  & 1,07  & 7  & 4,91  \\ \hline
    -3  & 1,4   & 8  & 5,30  \\ \hline
    -2  & 1,75  & 9  & 5,66  \\ \hline
    -1  & 2,1   & 10 & 6,04  \\ \hline
    \end{tabular}
    \end{center}
\end{table}

\noindentПо МНК посчитал коэф-т пропорциональности из соотношения: $ m = \tau z + b $. График
зависимости представлен на Рисунке 2.

\pic{0.6\linewidth}{graph1.jpg}{m(z)}

\noindentПолучил: $ \tau = 2,87 \pm 0,02 \texttt{мм}^{-1} $

\subsection{Зависимость показателя преломления от давления}

Изменяя давление с помощью сильфона и совмещая нулевые полосы получил зависимость n(P).
При этом: $ \Delta n = \tau\left( z - z(0) \right) \frac{\lambda}{L} $ .Данные представлены
в Таблице 2.

\begin{table}[h!]
    \begin{center}
    Таблица 2. \\
    \begin{tabular}{|l|l|l|l|}
    \hline
    $ \Delta P $, мм рт. с. & $ \Delta P $, Па  & z, мм & $ \Delta n, 10^{-6} $ \\ \hline
    -1000 & -10000 & 6,21 & 27,00  \\ \hline
    -600  & -6000  & 4,77 & 17,09  \\ \hline
    -200  & -2000  & 3,22 & 5,64   \\ \hline
    0     & 0      & 2,46 & 0,00   \\ \hline
    300   & 3000   & 1,65 & -6,18  \\ \hline
    600   & 6000   & 0,81 & -12,59 \\ \hline
    800   & 8000   & 0,23 & -17,24 \\ \hline
    \end{tabular}
    \end{center}
\end{table}

\newpage

\noindentЗависимость представлена на Рисунке 3.

\pic{0.6\linewidth}{graph2}{$\Delta n(z)$}

\noindentРассчитал коэф-т пропорциональности $ \alpha = \frac{kT}{2\pi} \cdot k\ruB{граф} =
    1,59 \pm 0,03 \cdot 10^{-30} \texttt{м}^{-3} $; \\

\noindentТогда: $ n = 1 + \frac{2\pi\alpha}{kT}P = 1,00025 \pm 0,00001 $. \\

\noindentЧто совпадает с таблицным значением в $ 1,00027 $.

\subsection{Зависимость показателя преломления С02 от времени}

Снял зависимость равновестного положения от времени, поминутно совмещая нулевые полосы.
Оценил время установления равновесия. Данные представлены в Таблице 3.

\begin{table}[h!]
    \begin{center}
    Таблица 3. \\
    \begin{tabular}{|l|l|l|l|l|l|}
    \hline
    $ \Delta Т $, мин & z, мм & $ n\ruB{газа} $ & $ \Delta Т $, мин & z, мм & $ n\ruB{газа} $ \\ \hline
    1 & 28,42 & 1,000467 & 10 & 13,32 & 1,000352 \\ \hline
    2 & 24,57 & 1,000438 & 11 & 12,89 & 1,000349 \\ \hline
    3 & 22,14 & 1,000419 & 12 & 12,12 & 1,000344 \\ \hline
    4 & 19,49 & 1,000399 & 13 & 11,86 & 1,000341 \\ \hline
    5 & 17,18 & 1,000381 & 14 & 11,35 & 1,000337 \\ \hline
    6 & 15,95 & 1,000372 & 15 & 11,17 & 1,000336 \\ \hline
    7 & 15,25 & 1,000367 & 16 & 10,94 & 1,000335 \\ \hline
    8 & 14,63 & 1,000362 & 17 & 10,85 & 1,000334 \\ \hline
    9 & 13,64 & 1,000355 & 18 & 10,76 & 1,000333 \\ \hline
    \end{tabular}
    \end{center}
\end{table}

\newpage

\noindentЗависимость представлена на Рисунке 4.

\pic{0.6\linewidth}{graph3.jpg}{$ n\ruB{газа}(\Delta T) $}

Время установления: $ \Delta T = 18 $ мин; \\

$ n_{CO_2} = 1,00046 \pm 0,00001 $. \\

Что совпадает с табличным значением в 1,00045.

\subsection{Интервал $ \Delta n $, доступный для измерения}

Оценю интервал $ \Delta n $, доступный для измерения: \\

\noindent$ \Delta n_{min} = m_{min} \frac{\lambda}{L} = 1 \cdot 10^{-6}$ \\

\noindent$ \Delta n_{max} = m_{max} \frac{\lambda}{L} = 25 \cdot 10^{-6}$

\section{Вывод}

В ходе данной работы интерферометр Релея был применен для вычисления показателей
преломеления воздуха и C02. Полученные значения совпали с табличными с хорошей точностью.

