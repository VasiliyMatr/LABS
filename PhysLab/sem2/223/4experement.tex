%!!! Experiment results and data here !!!%

\section{\Large Ход работы}

\begin{center}
    {\Large\bf Подготовка к эксперименту}
\end{center}

\subsection{Предварительные расчеты}
\hspace{1cm}Провел предварительные расчёты параметров опыта. Приняв максимально допустимый перегрев нити относительно термостата равным \mth{\Delta t_{max} = 10^\circ C}, оценил максимальную мощность нагрева \mth{Q_{max}[\text{мВт}],} которую следует подавать на нить. Для этой оценки принял коэффициент теплопроводности воздуха \mth{k \,\, \mathtt{\sim} \,\, 25 \frac{\text{мВт}}{\text{м}\cdot K}} \\

\noindent\hspace{1cm}Зная приблеженное значение сопростивления нити R, определил  соответствующие значения максимального тока \mth{I_{max}} и максимального напряжения \mth{U_{max}} \\

\noindent\hspace{1cm}{\bfПолученные значения:} \\[0.2cm]

\noindent\mth{Q_{max} = 102,9 \,\text{мВт}}\\

\noindent\mth{R = 13,0 \,\text{Ом}}\\

\noindent\mth{I_{max} = 89,0 \,\text{мА}}\\

\noindent\mth{U_{max} = 1,16 \,\text{В}}\\

\subsection{Подготовил эксперементальную установку к работе:}

\hspace{1.2cm}\begin{minipage}[t]{15cm}
\begin{itemize}
    \item Проверил, что измерительная схема собрана правильно;
    
    \item На магазине сопротивлений (или на реостате) установил максимальное сопростивление \mth{R\ruB{м}} ((чтобы ток в цепи при её замыкании был минимален);
    
    \item Включил вольтметр и амперметр и настроил режимы их работы (по техническому описанию к установке);

    \item Включил источник питания; проверил, что он работает в режиме источника напряжения, и что напряжение на нём не превышает максимально допустимое (указано на установке);
    
    \item Включил термостат и убедился, что вода в нём находится при комнатной температуре (измеренной по комнатному термометру).
    
\end{itemize}
\end{minipage}

\newpage

\begin{center}
    {\Large\bf Проведение измерений}
\end{center}

\subsection{Измерение зависимости R(Q)}
Провел серии экспериментов при разных температурах. \mth{U\ruB{э}} - Напряжение на эталонном сопротивлении; \mth{U\ruB{н}} - Напряжение на нити; R - Расчетное сопротивление нити; Q - Расчетная мощность на нити. Результаты представлены в таблице 1:

    \begin{table}[h!]
    	\begin{center}
    		\caption*{\color[HTML]{000000}Таблица 1: Измерение зависимости R(Q)}
    		\begin{tabular}{||P{3cm}|P{1.3cm}|P{1.3cm}|P{1.3cm}|P{1.3cm}|P{1.3cm}|P{1.3cm}|P{1.3cm}|P{1.3cm}||}
    	
    			\hline
    			\multicolumn{9}{||c||}{t = \mth{25,4^\circ} C} \\
    			\hline
    			
    			U\mth{\ruB{э}}, мВ & 50,0 & 75,0 & 100,0 & 200,0 & 350,0 & 500,0 & 700,0 & 850,0 \\
    			\hline
    			U\mth{\ruB{н}}, мВ & 62,5 & 93,8 & 124,9 & 249,8 & 438,9 & 628,8 & 885,4 & 1081,7 \\
    			\hline
    			R, Ом   & 12,56 & 12,51 & 12,49 & 12,49 & 12,54 & 12,58 & 12,66 & 12,73 \\
    			\hline
    			Q, мВт & 0,31 & 0,70 & 1,25 & 5,00 & 15,36 & 31,44 & 62,01 & 91,94\\

                \hline
    			\hline
    			\multicolumn{9}{||c||}{t = \mth{30,2^\circ} C} \\
    			\hline
    			
    			U\mth{\ruB{э}}, мВ  & 50,0 & 75,0 & 100,0 & 200,0 & 350,0 & 500,0 & 700,0 & 850,0 \\
    			\hline
    			U\mth{\ruB{н}}, мВ  & 63,7 & 95,7 & 127,6 & 255,5 & 447,8 & 642,1 & 903,2 & 1104,0 \\
    			\hline
    			R, Ом   & 12,74 & 12,76 & 12,76 & 12,78 & 12,79 & 12,84 & 12,90 & 12,99 \\
    			\hline
    			Q, мВт & 0,32 & 0,72 & 1,28 & 5,11 & 15,67 & 32,10 & 63,22 & 93,84 \\

    			\hline
    			\hline
    			\multicolumn{9}{||c||}{t = \mth{40,1^\circ} C} \\
    			\hline
    			
    			U\mth{\ruB{э}}, мВ  & 50,0 & 75,0 & 100,0 & 200,0 & 350,0 & 500,0 & 700,0 & 840,0 \\
    			\hline
    			U\mth{\ruB{н}}, мВ  & 65,6 & 98,4 & 131,1 & 262,5 & 460,2 & 659,5 & 928,2 & 1120,1 \\
    			\hline
    			R, Ом   & 13,12 & 13,12 & 13,11 & 13,13 & 13,15 & 13,19 & 13,26 & 13,33 \\
    			\hline
    			Q, мВт & 0,33  & 0,74  & 1,31  & 5,25 & 16,11 & 32,98 & 64,97 & 94,09 \\
    			
    			
                \hline
    			\hline
    			\multicolumn{9}{||c||}{t = \mth{50,4^\circ} C} \\
    			\hline
    			
    			U\mth{\ruB{э}}, мВ  & 50,0 & 75,0  & 100,0 & 200,0 & 350,0 & 500,0 & 700,0 & 800,0 \\
    			\hline
    			U\mth{\ruB{н}}, мВ  & 67,5 & 101,2 & 135,0 & 270,4 & 474,0 & 679,1 & 956,3 & 1096,2 \\
    			\hline
    			R, Ом   & 13,50 & 13,49 & 13,50 & 13,52 & 13,55 & 13,58 & 13,66 & 13,70 \\
    			\hline
    			Q, мВт & 0,34  & 0,76  & 1,35  & 5,41 & 16,60 & 33,95 & 66,94 & 87,70 \\
    
                \hline			
    			\hline
    			\multicolumn{9}{||c||}{t = \mth{60,4^\circ} C} \\
    			\hline
    			
    			U\mth{\ruB{э}}, мВ  & 50,0 & 75,0  & 100,0 & 200,0 & 350,0 & 500,0 & 700,0 & 800,0 \\
    			\hline
    			U\mth{\ruB{н}}, мВ  & 69,3 & 104,0 & 138,6 & 277,4 & 486,4 & 697,0 & 981,0 & 1125,0 \\
    			\hline
    			R, Ом   & 13,86 & 13,87 & 13,86 & 13,87 & 13,90 & 13,94 & 14,01 & 14,06 \\
    			\hline
    			Q, мВт & 0,35  & 0,78  & 1,39  & 5,55 & 17,02 & 34,85 & 68,67 & 90,00 \\
    			
                \hline
    		\end{tabular}
    	\end{center}
    \end{table}
    
\newpage