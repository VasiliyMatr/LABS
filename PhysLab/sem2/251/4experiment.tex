
\section{Ход работы}

\subsection{Проверка установки}
Проверил герметичность установки - успешно.

\subsection{Подбор частоты}
Подобрал частоту падения капель из аспиратора так, чтобы максимальное давление манометра не зависело от этой частоты (не чаще, чем 1 капля в 5 секунд).

\subsection{Максимальное давление спирта}

Измерил максимальное давление \mth{\Delta P\ruB{спирт}}  при  пробулькивании пузырьков воздуха через спирт.Пользуясь табличным значением коэффициента поверхностного натяжения спирта, определил по формуле (\ref{Laplas}) диаметр иглы. \\[0.2cm]

\noindent\mth{\Delta P\ruB{спирт} = 0000 \text{Па}} \\

\noindent\mth{\sigma^{\Delta P\ruB{спирт}} \ruB{случ} = 0000 \text{Па}} \\

\noindent\mth{d\ruB{иглы} = 0000 \text{мм}}

\subsection{Максимальное давление для поверхности воды}

Перенес предварительно промытую и просушенную от спирта иглу в колбу с дистиллированной водой. Измерил максимальное давление Р1 при пробулькивании пузырьков, когда игла лишь касается поверхности воды. Измерил расстояние между верхним концом иглы и !!! любой неподвижной частю !!! прибора h1. \\[0.2cm]

\noindent\mth{P1 = 0000\text{Па}} \\

\noindent\mth{h1 = 0000\text{м}}

\subsection{Максимальное давление на предельной глубине}

Утопил иглу до предела (между концом иглы и дном оставил небольшой зазор, чтобы образующийся пузырёк не касался дна). Измерил h2 (как в пункте 4).
Измерил максимальное давление в пузырьках Р2. По разности давлений \mth{\Delta P = P2 - P1} определил глубину погружения \mth{\Delta h} иглы и сравнил с \mth{\Delta h =  h1- h2}. 

\newpage

\mth{h2 = 0000\text{м}} \\

\mth{P2 = 0000\text{Па}} \\

\mth{\Delta P = 0000\text{Па}} \\

\mth{\Delta h = !!! = 0000\text{м}} \\

\mth{\Delta h = h1 - h2 = 0000\text{м}} \\

Значения \mth{\Delta h}, посчитаные разными способами, совпадают в пределах погрешностей.

\subsection {Зависимость $\sigma(T)$}

Снял зависимость $\sigma(T)$, результаты представлены в таблице 1.

\begin{table}[h!]
    \begin{center}
        \caption*{\color[HTML]{000000}Таблица 1: Зависимость $\sigma(T)$}
        \begin{tabular}{|| P{1.3cm}|P{1.3cm} || P{1.3cm}|P{1.3cm} || P{1.3cm}|P{1.3cm} || P{1.3cm}|P{1.3cm} ||} 
        \hline
        \hline

        \multicolumn{2}{||c||}{$ t = 25^\circ C$} & \multicolumn{2}{c||}{$ t = 30^\circ C$} & \multicolumn{2}{c||}{$ t = 35^\circ C$} & \multicolumn{2}{c||}{$ t = 40^\circ C$} \\

        \hline 
        
        P, Па & $\sigma, \frac{H}{m}$ & P, Па & $\sigma, \frac{H}{m}$ & P, Па & $\sigma, \frac{H}{m}$ & P, Па & $\sigma, \frac{H}{m}$ \\

        \hline

        0 & 0 & 0 & 0 & 0 & 0 & 0 & 0 \\
        \hline
        0 & 0 & 0 & 0 & 0 & 0 & 0 & 0 \\
        \hline

        \end{tabular}

        \vspace{0.5cm}

        \begin{tabular}{|| P{1.3cm}|P{1.3cm} || P{1.3cm}|P{1.3cm} || P{1.3cm}|P{1.3cm} ||} 
            \hline
            \hline

            \multicolumn{2}{||c||}{$ t = 45^\circ C$} & \multicolumn{2}{c||}{$ t = 50^\circ C$} & \multicolumn{2}{c||}{$ t = 55^\circ C$} \\

            \hline 

            P, Па & $\sigma, \frac{H}{m}$ & P, Па & $\sigma, \frac{H}{m}$ & P, Па & $\sigma, \frac{H}{m}$ \\

            \hline

            0 & 0 & 0 & 0 & 0 & 0 \\
            \hline
            0 & 0 & 0 & 0 & 0 & 0 \\
            \hline

            \end{tabular}

    \end{center}
\end{table}
