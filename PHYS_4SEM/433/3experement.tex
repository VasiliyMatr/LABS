
\section{Ход работы}

Длинна волны лазера: $ \lambda = 532 \texttt{нм} $. \\
Расстояние от сетки до экрана: $ L = (127 \pm 1) \texttt{см} $.

\subsection{Определение периода решеток по их пространственному спектру}

Собрал установку. Вращая наружное кольцо кассеты, получил дифракционные картины для
разных решеток. Измерил расстояния между соседними дифракционными максимумами. Данные
представлены в таблице 1 (гор.) и таблице 2 (верт.).

\begin{table}[h!]
    \begin{center}
    Таблица 1. \\
    \begin{tabular}{|l|l|l|l|l|l|}
    \hline
                 & 1    & 2    & 3    & 4    & 5    \\ \hline
    N            & 6    & 10   & 8    & 16   & 15   \\ \hline
    l            & 22,1 & 24,5 & 10,0 & 9,8  & 6,9  \\ \hline
    $ \Delta l $ & 3,68 & 2,45 & 1,25 & 0,61 & 0,46 \\ \hline
    \end{tabular}
    \end{center}
\end{table}

\begin{table}[h!]
    \begin{center}
    Таблица 2. \\
    \begin{tabular}{|l|l|l|l|l|l|}
    \hline
                 & 1    & 2    & 3    & 4    & 5    \\ \hline
    N            & 6    & 10   & 8    & 16   & 16   \\ \hline
    l            & 22,0 & 24,5 & 9,9  & 9,8  & 7,5  \\ \hline
    $ \Delta l $ & 3,67 & 2,45 & 1,24 & 0,61 & 0,47 \\ \hline
    \end{tabular}
    \end{center}
\end{table}

По полученным данным рассчитал периоды решеток.
$ \sigma_d = d\frac{\sigma{\Delta l}}{l} $
Данные предоставлены в таблице 3.

\begin{table}[h!]
    \begin{center}
    Таблица 3. \\
    \begin{tabular}{|l|l|l|l|l|l|}
    \hline
                      & 1    & 2    & 3    & 4     & 5     \\ \hline
    d, мкм            & 18,4 & 27,6 & 54,5 & 110,8 & 143,8 \\ \hline
    $ \sigma d $, мкм & 0,5  & 1,1  & 4,4  & 18,2  & 30,1  \\ \hline

    \end{tabular}
    \end{center}
\end{table}

\subsection{Определение периода решеток по изображению, увеличенному с помощью модели микроскопа}

Собрал модель проекционного микроскопа. Параметры установки:

\noindent$ a1 = 6,0 $ см \\
\noindent$ b1 = 51,0 $ см \\
\noindent$ a2 = 5,0 $ см \\
\noindent$ b2 = 57,5 $ см

Периоды сетки представлены в таблице 4.

\begin{table}[h!]
    \begin{center}
    Таблица 4. \\
    \begin{tabular}{|l|l|l|l|l|l|}
    \hline
          & 1    & 2    & 3    & 4     & 5     \\ \hline
    N     & 24   & 20   & 10   & 10    & 10    \\ \hline
    l, см & 4,2  & 5,3  & 5,5  & 10,5  & 14,3  \\ \hline

    \end{tabular}
    \end{center}
\end{table}

Посчитаю: $ \texttt{Г} = \frac{b1b2}{a1a2} = 96,9 $

Тогда посчитаю $ d = \frac{\Delta l}{\texttt{Г}} $. Результаты представлены в таблице 5.

\begin{table}[h!]
    \begin{center}
    Таблица 5. \\
    \begin{tabular}{|l|l|l|l|l|l|}
    \hline
                      & 1    & 2    & 3    & 4     & 5     \\ \hline
    d, мкм            & 18,1 & 27,3 & 56,7 & 108,4 & 147,6 \\ \hline
    $ \sigma d $, мкм & 0,5  & 1,1  & 4,5  & 18,1  & 30,3  \\ \hline

    \end{tabular}
    \end{center}
\end{table}

\subsection{Определение периода решеток по оценке разрешающей сопсобности микроскопа}

Поместил щелевую диафрагму с микрометрическим винтом в фокальную плоскость F линзы Л1.
Определил для каждой сетки минимальный размердиафрагмы D, при котором на экране еще
видно изображение сетки. Данные представлены в таблице 6.

\begin{table}[h!]
    \begin{center}
    Таблица 6. \\
    \begin{tabular}{|l|l|l|l|l|l|}
    \hline
                      & 1    & 2    & 3    & 4     & 5     \\ \hline
    D, мкм            & > 4  & > 4  & 2,0  & 1,0   & 0,8   \\ \hline
    d, мкм            & ---  & ---  & 57,7 & 112,6 & 167,6 \\ \hline
    $ \sigma d $, мкм & ---  & ---  & 12,0 & 19,3  & 32,0  \\ \hline

    \end{tabular}
    \end{center}
\end{table}

\subsection{Пространственная фильтрация и мультиплицирование}

Получил необходимые изображения:

\pic{0.4\linewidth}{pattern1.jpg}{изображение 1}
\pic{0.4\linewidth}{pattern2.jpg}{изображение 2}
\pic{0.4\linewidth}{pattern3.jpg}{изображение 3}

\newpage

\section{Вывод}

Полученные разными способами значения d совпадают в пределах погрешности, что
подтверждает теоретические сведения на практике.
