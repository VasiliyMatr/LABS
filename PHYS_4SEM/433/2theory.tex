
% Theory part goes here %

% for numerated formulas
\newcommand{\formula}[2]
{
    \begin{equation}\label{#1}
        #2
    \end{equation}
}

% for in-text math formulas
\newcommand{\mth}[1]
{
    \begin{math}
        #1
    \end{math}
}

% for rus letters in indexes
\newcommand{\ruB}[1]
{
    _{\text{#1}}
}

\newcounter{PicsCounter}
\setcounter{PicsCounter}{1}

\newcommand{\pic}[3]{
    \begin{center}
    \begin{minipage}[h!]{#1}
    \begin{center}

    \includegraphics[width = \textwidth]{#2}
    \textit{Рис \arabic{PicsCounter}. #3}

    \end{center}
    \end{minipage}
    \end{center}

    \stepcounter{PicsCounter}
}

\section{Теория}

Разрешающей способностью оптического прибора называется минимальное расстояние
$ l_{min} $ между двумя точками предмета, изображения которых разрешаются методом Релея.

\subsection{Подход Аббе к нахождению разрешающей способности микроскопа}

Изображение в фокальной плоскости F - источник вторичных волн. Из этих волн возникает
вторичное изображение. \\

\noindentПервичное изображение будет представлять собой дифракицонную картину Фраунгофера. Тогда
для решетки с периодом d направления максимальной интенсивности $ \phi_m $ определяются
с помощью формулы:

\formula
{phi-m}
{d \sin\phi_m = m\lambda} \\

\noindentПри этом, так как линза конечна, будут наблюдаться дифракционные искажения.

\noindentЕсли приоткрыть диафрагму, то будет наблюдаться переодическая картина.

\noindentУсловие разрешения решетки с периодом d:

\formula
{resolution}
{d \geq \frac{\lambda}{2\sin u},}

где u - апертурный угол.

\section{Схема установки}

Схема установки представлена на рисунке 1. \\

\pic{0.8\linewidth}{scheme.jpg}{Схема установки}

\newpage
