
% Theory part goes here %

% for numerated formulas
\newcommand{\formula}[2]
{
    \begin{equation}\label{#1}
        #2
    \end{equation}
}

% for in-text math formulas
\newcommand{\mth}[1]
{
    \begin{math}
        #1
    \end{math}
}

% for rus letters in indexes
\newcommand{\ruB}[1]
{
    _{\text{#1}}
}

\newcounter{PicsCounter}
\setcounter{PicsCounter}{1}

\newcommand{\pic}[3]{
    \begin{center}
    \begin{minipage}[h!]{#1}
    \begin{center}

    \includegraphics[width = \textwidth]{#2}
    \textit{Рис \arabic{PicsCounter}. #3}

    \end{center}
    \end{minipage}
    \end{center}

    \stepcounter{PicsCounter}
}

\section{Теория}

Усиление активного элемента в установившемся режиме за один проход:
\formula
{increase}
{G = \frac{1}{\sqrt{R_1R_2}T^2}}

Обычно достигается усиление в 1-3\% за проход, то есть $ G = 1.01 - 1.03 $. \\ [0.1cm]

\noindentШирина спектра генерации:
\formula
{spectrumWidth}
{\Delta\nu = 2\nu\sqrt{\frac{2kT\ln{2}}{mc^2}}}

Получаем оценку для ширины спектра: $ \Delta\nu \approx 1 \texttt{Гц} $.

\section{Схема установки}

Схема установки представлена на рисунке 1. \\ [0.2cm]
\pic{0.9\linewidth}{scheme.jpg}{Схема установки}

\newpage
