
\section{Ход работы}

\subsection{Коэффициент усиления лазера}

Настроили установку для данного пункта. Сигнал с фотодиода (6) идет на канал (1).
Сигнал с фотодиода (11) идет на канал (2). Произвели измерения напряжения при
включенном и выключенном источнике тока. \\ [0.1cm]

\noindentКоэффициент k расчитывается по формуле:
\formula
{increase-exp}
{k = \frac{\alpha\ruB{вкл}}{\alpha\ruB{выкл}},}

где $ \alpha\ruB{вкл} = \frac{U\ruB{вкл1}}{U\ruB{вкл2}} $,
$ \alpha\ruB{выкл} = \frac{U\ruB{выкл1}}{U\ruB{выкл2}} $ \\ [0.1cm]

\noindentРезультаты измерений представлены в Таблице 1.

\begin{table}[h!]
\begin{center}
    Таблица 1: \\
    \begin{tabular}{|l|l|l|l|l|l|l|l|}
    \hline
    I  & U    & U    & U    & U    & a     & a     & k     \\ \hline
    20 & 35,5 & 31,3 & 37,1 & 33,8 & 1,135 & 1,098 & 1,033 \\ \hline
    38 & 39,8 & 36,0 & 39,1 & 36,0 & 1,106 & 1,087 & 1,018 \\ \hline
    46 & 39,7 & 33,9 & 30,2 & 28,3 & 1,171 & 1,066 & 1,098 \\ \hline
    \end{tabular}
\end{center}
\end{table}

\subsection{Зависимость интенсивности от угла поворота поляроида}

Закрепили в рейтере перед выходным зеркалом поляроид. Юстировочный лазер отключили.
Измерили зависимость интенсивности излучения исследуемого лазера в зависимости от угла
поворота поляроида. Результаты представлены в Таблице 2.

\begin{table}[h!]
    \begin{center}
    Таблица 2: \\
    \begin{tabular}{|l|l|l|l|l|l|l|l|}
    \hline
    a  & U     & a  & U     & a   & U    & a   & U     \\ \hline
    0  & 26,70 & 50 & 22,53 & 100 & 0,53 & 150 & 13,51 \\ \hline
    10 & 28,75 & 60 & 16,86 & 110 & 0,16 & 160 & 17,09 \\ \hline
    20 & 30,16 & 70 & 12,37 & 120 & 1,89 & 170 & 22,94 \\ \hline
    30 & 29,60 & 80 & 8,80  & 130 & 3,64 & 180 & 27,06 \\ \hline
    40 & 25,17 & 90 & 4,17  & 140 & 8,86 &     &       \\ \hline
    \end{tabular}
    \end{center}
\end{table}

\newpage

\noindentПостоил график $ U(\alpha) $. График предоставлен на рисунке 2.

\pic{0.8\linewidth}{graph.jpg}{Зависимость $ U(\alpha) $}

\noindentИз графика видно, что закон Малиса выполняется, так как $ U(\alpha) $ является
гармонической функцией с $ T = \pi $

\subsection{Наблюдение модовой структуры лазерного излучения}

Получили различные режимы модовой структуры. См рисунки 3-6.

\pic{0.6\linewidth}{mode1.jpg}{Одна мода}
\pic{0.6\linewidth}{mode2.jpg}{Две моды}

\pic{0.6\linewidth}{mode3.jpg}{Три моды}
\pic{0.6\linewidth}{mode4.jpg}{Много мод}

\newpage

\section{Вывод}

В ходе данной лабараторной работы был измерен коэффициент усиления лазера, значение
которого по порядку совпало с теоретическим. Так же была подтверждена формула Малиса.
И наблюдались изменения модовой структуры излучения в зависимости от отклонений зеркал
лазера.