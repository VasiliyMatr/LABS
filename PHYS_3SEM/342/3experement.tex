
\section{Ход работы}

\subsection{Подготовил приборы к работе}

Оценил допустимую ЭДС термопары: $dV = k * \triangle T = 12$ мВ, где $k = 24$град/мВ и $\triangle T = 0.5 ^{\circ}$С.

\subsection{Исследовал проверяемую зависимость}

Исследовал зависимость периода колебания генератора от темературы образца, отмечая период колебаний $\tau$ по частотомеру, а температуру $T$ - по показаниям цифрового вольтметра. Снятые значения представлены в таблице 1.

\begin{table}[h!]
\center
\begin{tabular}{|l|l|l|l|l|}
\hline
$t, ^\circ C$ & $\tau, \text{мкс}$ & $T, \text{К}$ & $\chi$ & $\frac{1}{\chi}$ \\ \hline
10,10 & 10,18 & 283,10 & 0,52 & 1,92 \\ \hline
10,85 & 10,17 & 283,85 & 0,52 & 1,93 \\ \hline
11,94 & 10,14 & 284,94 & 0,51 & 1,96 \\ \hline
12,90 & 10,11 & 285,90 & 0,50 & 2,00 \\ \hline
14,87 & 10,02 & 287,87 & 0,47 & 2,11 \\ \hline
16,75 & 9,88  & 289,75 & 0,43 & 2,30 \\ \hline
18,72 & 9,64  & 291,72 & 0,36 & 2,75 \\ \hline
20,75 & 9,23  & 293,75 & 0,25 & 4,00 \\ \hline
22,70 & 8,88  & 295,70 & 0,16 & 6,29 \\ \hline
24,69 & 8,68  & 297,69 & 0,11 & 9,46 \\ \hline
26,68 & 8,58  & 299,68 & 0,08 & 12,53\\ \hline
\end{tabular}
\end{table}

\noindentТакже данные представлены на рисунке 2.

\pick{0.8\textwidth}{picks/graph.png}{График $1/\chi (t)$}

\newpage

\noindentПо графику получаю значение точки Кюри для Гадолиния: \\
$\Theta_p = 20\pm1 ^\circ C$.

\section{Вывод}

В ходе данной лабараторной работы закон Кюри-Вейсса был эксперементально подтвержден. Так же полученное значение точки Кюри для гадолиния совпало с табличным в пределах погрешности.
