
% Theory part goes here %

% for numerated formulas
\newcommand{\formula}[2]
{
    \begin{equation}\label{#1}
        #2
    \end{equation}
}

% for in-text math formulas
\newcommand{\mth}[1]
{
    \begin{math}
        #1
    \end{math}
}

% for rus letters in indexes
\newcommand{\ruB}[1]
{
    _{\text{#1}}
}

\newcounter{PicsCounter}
\setcounter{PicsCounter}{1}

\newcommand{\pic}[3]{
    \begin{center}
        \begin{minipage}[h!]{#1}
            \begin{center}

                \includegraphics[width = \textwidth]{#2}
                \textit{Рис \arabic{PicsCounter}. #3}

            \end{center}
        \end{minipage}
    \end{center}

    \stepcounter{PicsCounter}
}

\newcounter{TablesCounter}
\setcounter{TablesCounter}{1}

\newcommand{\tableLable}[1]{
    \textit{Таблица \arabic{TablesCounter}: #1}

    \stepcounter{TablesCounter}
}

\section{Теория}

\textbf{В работе используются:} оптический пирометр; модель абсолютно чёрного тела; вольфрамовая
лампа; неоновая лампа; блок питания; цифровые вольтметр и амперметр; термопара. \\

В работе измеряется яркостная температура. \textbf{Яркостная температура} - это
температура абсолютно чёрного тела, при которой его спектральная испускательная
способность равна спектральной испускательной способности исследуемого тела при той же
длине волны. \\

Зависимость термодинамической температуры от яркостной температуры для вольфрама
представлена на рисунке 1.

\pic{0.8\linewidth}{graph1.jpg}
{Зависимость термодинамической температуры от яркостной температуры для вольфрама}

По результатам измерений мощности излучения вольфрамовой нити можно судить о
справедливости закона Стефана-Больцмана. Если бы нить излучала как АЧТ, то баланс
потребляемой и излучаемой энергии определялся бы соотношением:

\formula{BasicStefanBoltzmann}
{ W = \sigma S (T^4 - T_0^4) }

Где $ W $ - потребляемая нитью электрическая мощность, $ S $ - площадь излучающей
поверхности нити, $ T $ - температура нити, $ T_0 $ - температура окружающей среды.

Для серого тела можно переписать в виде:

\formula{TransformedStefanBoltzmann}
{ W = \varepsilon_T S \sigma T^4 }

В справедливости закона Стефана-Больцмана можно убедиться, построив график зависимости
$ W\left( T \right) $ в логарифмическом масштабе и по углу наклона определить показатель
степени $ n $ исследуемой температурной зависимости. В пределах погрешности показатель
степени должен быть близок к четырём.
