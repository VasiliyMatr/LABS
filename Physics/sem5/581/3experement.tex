
% Experement data + calcs goes here %

\section{Ход работы}

\subsection{Изучение работы оптического пирометра}

Проверили, что различия в значениях термодинамической и яркостной температуры АЧТ
менее $ 10\% $.

\subsection{Измерение яркостной температуры тел}

Убедились, что различные тела, нагретые до одинаковой термодинамической температуры
могут иметь различную яркостную температуру - использовали нагретые кольца с различными
к-тами отражения. Данную проверку получилось провести только посредством визуального
наблюдения - яркостная температура некоторых колец была слишком мала для измерения с
помощью пирометра. Фотография колец предоставлена на рисунке 2.

\pic{0.8\linewidth}{rings.jpg}{Наблюдаемые колца}

\subsection{Проверка закона Стефана-Больцмана}

Постепенно меняя накал нити лампы в диапазоне $ 900-1800^\circ C $ измерили
пирометром яркостную температуру нити накала лампы, а так же значение силы тока
и напряжения на ней. Определили так же по значениям яркостной температуры нити ее
термодинамическую температуру, используя зависимость $ T \left( T\ruB{ярк} \right) $
(см. рисунок 1). \\

Полученные значения представлены в таблице 1.

\begin{table}[h!]
\begin{center}

    \tableLable{Данные}
    \begin{tabular}{|c|c|c|c|c|c|c|c|c|c|c|}
    \hline
    $ t, ^\circ C $          & 900   & 1000  & 1100  & 1200  & 1300  & 1400  & 1500  & 1600  & 1700  & 1800  \\ \hline
    $ t_{term} \, ^\circ C $ & 920   & 1030  & 1140  & 1250  & 1350  & 1450  & 1550  & 1660  & 1770  & 1880  \\ \hline
    $ I, \, $ А              & 0,684 & 0,761 & 0,851 & 0,961 & 1,056 & 1,148 & 1,227 & 1,352 & 1,450 & 1,548 \\ \hline
    $ V, \, $ В              & 22,3  & 28,2  & 35,7  & 45,8  & 54,7  & 64,5  & 73,1  & 87,4  & 99,4  & 105,4 \\ \hline
    $ W, \, $ Вт             & 15.3  & 21.5  & 30.4  & 44.0  & 57.8  & 74.0  & 89.7  & 118.2 & 144.1 & 163.2 \\ \hline
    \end{tabular}

\end{center}
\end{table}

Построил зависимость $ W \left( T \right) \, $ в логарифмическом масштабе. Зависимость
предоставлена на рисунке 3.

\pic{0.8\linewidth}
{graph2.jpg}
{Зависимость $ W \left( T \right) \, $ в логарифмическом масштабе}

Получил коэффициент наклона: \\

$ n = 3,43 \pm 0,7 $ \\

Посчитал постоянную Стефана-Больцмана для $ 1700 ^\circ C \, $ и $ 1800 ^\circ C \, $: \\

$ \sigma_{1700} = \left( 3,7 \pm 0,9 \right) \cdot 10^{-5} \,
\frac{\text{эрг}} {\text{с} \cdot \text{см}^2 \cdot \text{К}^4} $ \\

$ \sigma_{1800} = \left( 3,9 \pm 0,9 \right) \cdot 10^{-5} \,
\frac{\text{эрг}} {\text{с} \cdot \text{см}^2 \cdot \text{К}^4} $ \\

\subsection{Измерение яркостной температуры неоновой лампы}

Термодинамическая температура лампы приблизительно равна комнатной. Однако ее яркостная
температура составляет $ \approx 980^\circ C $. Данное расхождение температур обусловлено
тем, что неоновая лампа не удолетворяет модели АЧТ.

\section{Вывод}

В ходе данной работы был проверен закон Стефана-Больцмана и границы его применимости. Успех!
Точность измерений можно повысить, заменив пирометр на более удобный (электронный).
