
% Theory part goes here %

% for numerated formulas
\newcommand{\formula}[2]
{
    \begin{equation}\label{#1}
        #2
    \end{equation}
}

% for in-text math formulas
\newcommand{\mth}[1]
{
    \begin{math}
        #1
    \end{math}
}

% for rus letters in indexes
\newcommand{\ruB}[1]
{
    _{\text{#1}}
}

\newcounter{PicsCounter}
\setcounter{PicsCounter}{1}

\newcommand{\pic}[3]{
    \begin{center}
        \begin{minipage}[h!]{#1}
            \begin{center}

                \includegraphics[width = \textwidth]{#2}
                \textit{Рис \arabic{PicsCounter}. #3}

            \end{center}
        \end{minipage}
    \end{center}

    \stepcounter{PicsCounter}
}

\newcounter{TablesCounter}
\setcounter{TablesCounter}{1}

\newcommand{\tableLable}[1]{
    \textit{Таблица \arabic{TablesCounter}: #1}

    \stepcounter{TablesCounter}
}

\section{Теория}

\subsection{Установка для измерения сечения рассеяния электронов в газах}

Схема установки представлена на рисунке 1.

\pic
{0.8\linewidth}
{scheme1.jpg}
{Установка для измерения сечения рассеяния электронов в газах}

\subsection{Теоретические выкладки}

Для налетающего электрона выполняется закон сохранения энергии:

\formula{EnergyConservationLaw}
{E = \frac{mv^2}{2} = \frac{mv^{'2}}{2} + U}

Следовательно, так как скорость электрона изменилась, то изменилась и длина его волны
де Бройля. Следовательно атом ведет себя как преломляющая среда. Относительный
показатель приломления:

\formula{RefractionFactor}
{n = \frac{\lambda}{\lambda^{'}} = \sqrt{1 - \frac{U}{E}}}

Уравнение Шредингера для потенциальной ямы:

\formula{SchrodingersEquation}
{\psi^{''} + k^2\psi}

При этом: $ k^2 = k_1^2 = \frac{2mE}{\hslash^2} $ для положения вне потенциальной ямы, и
$ k^2 = k_2^2 = \frac{2m\left( E + U_0 \right)}{\hslash^2} $ для положения в области
потенциальной ямы. \\

Коэффициент прохождения:

\formula{TransmissionCoefficient}
{
    D = \frac{16k_1^2k_2^2}
        {16k_1^2k_2^2 + 4\left( k_1^2 - k_2^2 \right)^2\sin^2\left( k_2l \right)}
}

Тогда видно, что коэффициент прохождения максимален при условии:

\formula{MaxTransmissionCoefficientCondition}
{k_2l = \sqrt{\frac{2m\left( E + U_0 \right)}{\hslash^2}}l = n\pi \, , \quad n = 1,2,3 \, ...}

Из рассуждений выше также получаем эффективный размер атома:

\formula{EffectiveAtomSize}
{l = \frac{h\sqrt{5}}{32m\left( E_2 - E_1 \right)}}

При этом $ E_1 = eV_1 \, $, и $ E_2 = eV_2 \, $. \\

Так же можно получить формулу для эффективной глубины потенциальной ямы атома:

\formula{EffectivePotentialWellDepth}
{U_0 = \frac{4}{5}E_2 = \frac{9}{5}E_1}
