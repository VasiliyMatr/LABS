
% Theory part goes here %

% for numerated formulas
\newcommand{\formula}[2]
{
    \begin{equation}\label{#1}
        #2
    \end{equation}
}

% for in-text math formulas
\newcommand{\mth}[1]
{
    \begin{math}
        #1
    \end{math}
}

% for rus letters in indexes
\newcommand{\ruB}[1]
{
    _{\text{#1}}
}

\newcounter{PicsCounter}
\setcounter{PicsCounter}{1}

\newcommand{\pic}[3]{
    \begin{center}
        \begin{minipage}[h!]{#1}
            \begin{center}

                \includegraphics[width = \textwidth]{#2}
                \textit{Рис \arabic{PicsCounter}. #3}

            \end{center}
        \end{minipage}
    \end{center}

    \stepcounter{PicsCounter}
}

\newcounter{TablesCounter}
\setcounter{TablesCounter}{1}

\newcommand{\tableLable}[1]{
    \textit{Таблица \arabic{TablesCounter}: #1}

    \stepcounter{TablesCounter}
}

\section{Теория}

Проходя через вещество, пучок $ \gamma $-квантов постепенно ослабляется, ослабление
происходит по экспоненциальному закону, который может быть записан в двух эквивалентных
формах: \\

\formula{eq::I_first}{
    I = I_0 e^{-\mu l}
}

\formula{eq::I_second}{
    I = I_0 e^{-\mu' m_l}
}

Где $ I, I_0 $ -- интенсивности прошедшего и падающего излучений, $ l $ -- длина пути,
пройденного пучком $ \gamma $-лучей, $ m_l $ -- масса пройденного вещества на единицу
площади, $ \mu $, $ \mu' $ -- константы, зависящие от вещества. Ослабление потока
$ \gamma $-лучей возникает из-за фотоэлектрического поглощения, комптоновского рассеяния и
генерации электрон-позитронных пар (при достаточных энергиях).\\

Считая, что опыт поставлен в \textit{хорошей геометрии}, то есть сквозь вещество всегда
идёт узкий параллельный пучок, можно считать, что комптоновское рассеяние выводит
$ \gamma $-кванты из пучка и в итоге меняется количество, но не энергия $ \gamma $-квантов.
Это означает, что $ \mu $ не зависит от $ l $. Число выбывших на пути $ dl $ из пучка
$ \gamma $-квантов: \\

\formula{eq::dN}{
    -dN = \mu N dl
}

Откуда: \\

\formula{eq::N}{
    N = N_0 e^{\mu l}
}

\formula{eq::mu} {
    \mu = \dfrac{1}{l} \ln \dfrac{N_0}{N}.
}

\section{Описание установки}

\pic{0.7\linewidth}{scheme1.jpg}{Схема установки}

На рис. 1 изображена схема установки. Свинцовый коллиматор выделяет узкий почти параллельный
пучок $ \gamma $-квантов, проходящий через набор поглотителей П и регистрируемый сцинтилляционным
счётчиком. Сигналы от счётчика усиливаются и регистрируются пересчётным прибором ПП.
Высоковольтный выпрямитель ВВ обеспечивает питание сцинтилляционного счётчика. Чтобы
уменьшить влияние плохой геометрии, счётчик расположен на большим расстоянии от источника,
поглотители имеют небольшие размеры, а так же устанавливаются на расстоянии друг от друга,
чтобы испытавшие комптоновское рассеяние кванты с меньшей вероятностью могли в него вернуться.

