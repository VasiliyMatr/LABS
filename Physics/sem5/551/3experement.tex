
% Experement data + calcs goes here %

\section{Ход работы}

\subsection{Настройка установки}

Убедились в том, что установка работает исправно - при полностью открытом коллиматоре и при
коллиматоре, закрытом свинцовой пробкой.

\subsection{Данные}

Сняли данные для различных толщин и материалов образцов. Данные представлены в таблицах 1
и 2.

\begin{table}[h!]
    \begin{center}

        \tableLable{Толщины образцов}
        \begin{tabular}{|l|c|c|c|c|c|}
        \hline
        Материал & l1, мм & l2, мм & l3, мм & l4, мм & l5, мм \\ \hline
        Алюминий & 19.80  & 19.95  & 19.80  & 19.95  & 20.00  \\ \hline
        Сталь    & 10.00  & 9.95   & 10.00  & 9.95   & 10.00  \\ \hline
        Свинец   & 4.80   & 4.90   & 4.70   & 4.60   & 4.4    \\ \hline
        \end{tabular}

    \end{center}
\end{table}

\begin{table}[h!]
    \begin{center}

        \tableLable{Число частиц в зависимости от образцов}
        \begin{tabular}{|l|c|c|c|c|c|}
        \hline
        Образец & Al 1   & Al 2   & Al 3  & Al 4  & Al 5  \\ \hline
        N       & 219904 & 139859 & 89422 & 58460 & 38075 \\ \hline
        \end{tabular} \\
        \begin{tabular}{|l|c|c|c|c|c|}
        \hline
        Образец & Fe 1   & Fe 2   & Fe 3  & Fe 4  & Fe 5  \\ \hline
        N       & 183792 & 99011  & 53681 & 30113 & 16662 \\ \hline
        \end{tabular} \\
        \begin{tabular}{|l|c|c|c|c|c|}
        \hline
        Образец & Pb 1   & Pb 2   & Pb 3  & Pb 4  & Pb 5  \\ \hline
        N       & 185379 & 98657  & 55228 & 30396 & 17977 \\ \hline
        \end{tabular}

    \end{center}
\end{table}

Графики в логарифмическом масштабе представлены на рисунках 2-4. \\

\pic{0.7\linewidth}{graph1.jpg}{График для Al}
\pic{0.7\linewidth}{graph2.jpg}{График для Fe}
\pic{0.7\linewidth}{graph3.jpg}{График для Pb}

По МНК определил коэффициенты наклона: \\

$ \mu_{Al} = \left( 0.22 \pm 0.01 \right) \frac{1}{cm} $ \\

$ \mu_{Fe} = \left( 0.60 \pm 0.03 \right) \frac{1}{cm} $ \\

$ \mu_{Pb} = \left( 1.26 \pm 0.08 \right) \frac{1}{cm} $ \\

Тогда определил значения энергии квантов излучения по таблице: \\

$ E_{\gamma Al} = 0.6 MeV $ \\

$ E_{\gamma Fe} = 0.6 MeV $ \\

$ E_{\gamma Pb} = 0.6 MeV $ \\

\section{Вывод}

В ходе данной работы было исследовано явление ослабления потока $ \gamma $-лучей в веществе.
Также были расчитаны значения постоянной затухания и энергия $ \gamma $-квантов.
