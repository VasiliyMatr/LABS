
% Experement data + calcs goes here %

\section{Схема установки}

Схема установки представлена на рисунке 1.

\pic{0.8\linewidth}{scheme1.jpg}{Схема установки}

\newpage

\section{Ход работы}

Подготовили установку к работе. Для углов больше $ 60^\circ \, $ необходимо будет
дополнительно учесть фоновые шумы. Они могут быть сняты отдельно при заблокированном
излучении образца. \\

Сняли зависимость номера канала от угла наблюдения. Так же учли фоновые шумы. График
$ 1/N \left[ 1 - \cos\left( \Theta \right) \right] $ представлен на рисунке 2.

\pic{0.8\linewidth}{graph1.jpg}{$ 1/N \left[ 1 - \cos\left( \Theta \right) \right] $}

Из аппроксимации получил наилучшие значения каналов для $ \Theta = 0^\circ \, $ и
$ \Theta = 90^\circ \, $: \\

$ N\ruB{наил} (0)  = \frac{1}{\frac{1}{N} (0)}  = 851 \pm 21 $ \\
$ N\ruB{наил} (90) = \frac{1}{\frac{1}{N} (90)} = 364 \pm 9 $ \\

Погрешности оценил исходя из следующих соотношений: \\

$ \sigma N = \sqrt{\sigma_{N_{quad}}^2 + N^2 \cdot \epsilon_{N_{mes}}^2} \, $ ;
$ \sigma N_{mes} = 10 $ \\

По формуле (\ref{ElectronRestEnergy}) рассчитал энергию покоя электрона: \\

$ m_ec^2 = \left( 495 \pm 30 \right) \, $ кэВ

\section{Вывод}

В ходе данной работы был проверен закон Комптона и было расчитано значение энергии покоя
электрона. Теоретические данные совпадают с эксперементальными. Точность измерений можно
существенно увеличить путем увеличения разрешения анализатора и уменьшения фоновых шумов.
