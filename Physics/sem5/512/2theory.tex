
% Theory part goes here %

% for numerated formulas
\newcommand{\formula}[2]
{
    \begin{equation}\label{#1}
        #2
    \end{equation}
}

% for in-text math formulas
\newcommand{\mth}[1]
{
    \begin{math}
        #1
    \end{math}
}

% for rus letters in indexes
\newcommand{\ruB}[1]
{
    _{\text{#1}}
}

\newcounter{PicsCounter}
\setcounter{PicsCounter}{1}

\newcommand{\pic}[3]{
    \begin{center}
        \begin{minipage}[h!]{#1}
            \begin{center}

                \includegraphics[width = \textwidth]{#2}
                \textit{Рис \arabic{PicsCounter}. #3}

            \end{center}
        \end{minipage}
    \end{center}

    \stepcounter{PicsCounter}
}

\newcounter{TablesCounter}
\setcounter{TablesCounter}{1}

\newcommand{\tableLable}[1]{
    \textit{Таблица \arabic{TablesCounter}: #1}

    \stepcounter{TablesCounter}
}

\section{Теория}

При рассеянии $ \gamma $-кванта на электроне выполняется соотношение Комптона:

\formula{ComptonEquation}
{ \Delta\lambda = \lambda_1 - \lambda_0 = \frac{h}{mc}\left( 1 - \cos{\Theta} \right) }

Где $ \lambda_0 \, $ и $ \lambda_1 \, $ - длинны волны $ \gamma $-кванта до и после
рассеяния соответственно; $ \Theta $ - угол рассеяния  $ \gamma $-кванта. \\

Формулу (\ref{ComptonEquation}) можно переписать в более удобном виде:

\formula{TransformedComptonEquation}
{ \frac{1}{\epsilon{\left( \Theta \right)}} - \frac{1}{\epsilon_0} = 1 - \cos{\Theta} }

Где $ \epsilon_0 \, $ - начальная энергия $ \gamma $-кванта в единицах $ mc^2 $;
$ \epsilon{\left( \Theta \right)} \, $ - энергия рассеянного $ \gamma $-кванта в единицах
$ mc^2 $. \\

Следует отметить, что формулы (\ref{ComptonEquation}) и (\ref{TransformedComptonEquation})
применимы только для свободных электронов, что справедливо для легких атомов, для которых
энергия связи мала в сравнении с энергией $ \gamma $-кванта. \\

Формулу (\ref{TransformedComptonEquation}) можно переписать использовав соотношение
$ \epsilon{\left( \Theta \right)} = AN{\left( \Theta \right)} \, $ :

\formula{ChannelComptonEquation}
{
    \frac{1}{N{\left( \Theta \right)}} - \frac{1}{N{\left( 0 \right)}} =
        A \left( 1 - \cos{\Theta} \right)
}

Тогда можно определить энергию покоя электрона:

\formula{ElectronRestEnergy}
{ mc^2 = E_\gamma \frac{N{\left( 90 \right)}} {N{\left( 0 \right)} - N{\left( 90 \right)}} }

Где $ E_\gamma \, $ - энергия испускаемых источником $ \gamma $-квантов. \\
$ E_\gamma = 662 $ кэВ