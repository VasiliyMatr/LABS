
% Theory part goes here %

% for numerated formulas
\newcommand{\formula}[2]
{
    \begin{equation}\label{#1}
        #2
    \end{equation}
}

% for in-text math formulas
\newcommand{\mth}[1]
{
    \begin{math}
        #1
    \end{math}
}

% for rus letters in indexes
\newcommand{\ruB}[1]
{
    _{\text{#1}}
}

\newcounter{PicsCounter}
\setcounter{PicsCounter}{1}

\newcommand{\pic}[3]{
    \begin{center}
        \begin{minipage}[h!]{#1}
            \begin{center}

                \includegraphics[width = \textwidth]{#2}
                \textit{Рис \arabic{PicsCounter}. #3}

            \end{center}
        \end{minipage}
    \end{center}

    \stepcounter{PicsCounter}
}

\newcounter{TablesCounter}
\setcounter{TablesCounter}{1}

\newcommand{\tableLable}[1]{
    \textit{Таблица \arabic{TablesCounter}: #1}

    \stepcounter{TablesCounter}
}

\section{Теория}

\subsection{Описание установки}

Ионизационная камера --- прибор для количественного измерения
ионизации, произведенной заряженными частицами при прохождении
через газ. Камера представляет собой наполненный газом сосуд с двумя электродами (схема камеры приведена на рис. \ref{ris Ion}). 

Заполняющий сосуд газ сам по себе не проводит электрический ток, возникает он только при прохождении быстрой заряженной частицы, которая рождает в газе на своем пути ионы.

Поместим на торец внутреннего электрода источник
ионизирующего излучения (в нашем случае это источник
альфа-частиц $ ^{239}_{94} $Pu), заполним объем камеры воздухом и начнем
постепенно увеличивать разность потенциалов между электродами. Ток, протекающий через камеру, вначале будет резко возрастать, а затем, начиная с некоторого напряжения $ V_0 $, станет постоянным, т. е. "<выйдет на плато">.  Предельный ток $ I_0 $ будет равен $ I_0 = n_0e $,
где $ n_0 $ --- число пар ионов, образуемых в секунду в объеме камеры, а $ e $ --- заряд электрона.

\pic
{0.8\linewidth}
{Ion.png}
{Схема устройства ионизационной камеры}

Прохождение тока через камеру регистрируется посредством измерения напряжения на включенном в цепь камеры сопротивлении $ R $.
Так как средняя энергия ионизации атомов воздуха составляет около 30 эВ, то альфа-частица с энергией 3 МэВ образует на своем пути около $ 10^5 $ электронов, им соответствует заряд $1,6 \cdot 10^{-14} $ Кл. Чтобы
столь малое количество заряда, создаваемое проходящей через камеру одной альфа-частицей, вызывало измеряемое напряжение, емкость $ C $
должна быть мала.

При изменении давления в камере ионизационный ток меняется сначала линейно, а потом выходит на насыщение. При небольших давлениях газа
альфа-частицы передают часть энергии стенкам камеры. По достижении
давления $ P_0 $ все они заканчивают свой пробег внутри газа, и дальнейшее возрастание тока прекращается.

В данной работе измерение пробега альфа-частицы проводится по величине тока ионизации в сферической камере. Вакуумная установка содержит кран и манометр. Она позволяет изменять давление в камере от атмосферного до 10 мм рт. ст.
Величина тока ионизации измеряется электрометром, состоящим из
нескольких стандартных микросхем, по величине падения напряжения
на сопротивлении $ R $ = 100 МОм ($ C = 10^{-8} $ Фарад, так что $ RC $ = 1 с).
Значение измеряемого ионизационного тока (в пикоамперах) высвечивается на цифровом табло.

\newpage
