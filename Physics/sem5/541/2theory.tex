
% Theory part goes here %

% for numerated formulas
\newcommand{\formula}[2]
{
    \begin{equation}\label{#1}
        #2
    \end{equation}
}

% for in-text math formulas
\newcommand{\mth}[1]
{
    \begin{math}
        #1
    \end{math}
}

% for rus letters in indexes
\newcommand{\ruB}[1]
{
    _{\text{#1}}
}

\newcounter{PicsCounter}
\setcounter{PicsCounter}{1}

\newcommand{\pic}[3]{
    \begin{center}
        \begin{minipage}[h!]{#1}
            \begin{center}

                \includegraphics[width = \textwidth]{#2}
                \textit{Рис \arabic{PicsCounter}. #3}

            \end{center}
        \end{minipage}
    \end{center}

    \stepcounter{PicsCounter}
}

\newcounter{TablesCounter}
\setcounter{TablesCounter}{1}

\newcommand{\tableLable}[1]{
    \textit{Таблица \arabic{TablesCounter}: #1}

    \stepcounter{TablesCounter}
}

\section{Теория}

\subsection{Ионизационная камера}

\pic
{0.5\linewidth}
{scheme1.jpg}
{Схема устройства ионизационной камеры}

Ионизационная камера --- прибор для количественного измерения ионизации, произведенной
заряженными частицами при прохождении через газ. Камера представляет собой наполненный
газом сосуд с двумя электродами. Схема камеры приведена на рисунке 1. \\

Заполняющий сосуд газ сам по себе не проводит электрический ток, возникает он только
при прохождении быстрой заряженной частицы, которая порождает в газе на своем пути ионы. \\

Поместим на торец внутреннего электрода источник ионизирующего излучения (в нашем случае
это источник альфа-частиц $ ^{239}_{94} Pu $), заполним объем камеры воздухом и начнем
постепенно увеличивать разность потенциалов между электродами. Ток, протекающий через
камеру, вначале будет резко возрастать, а затем, начиная с некоторого напряжения $ V_0 $,
станет постоянным. Предельный ток $ I_0 $ будет равен $ I_0 = n_0e $, где $ n_0 $ ---
число пар ионов, образуемых в секунду в объеме камеры, а $ e $ --- заряд электрона. \\

Прохождение тока через камеру регистрируется посредством измерения напряжения на включенном
в цепь камеры сопротивлении $ R $. Так как средняя энергия ионизации атомов воздуха
составляет около 30 эВ, то альфа-частица с энергией 3 МэВ образует на своем пути около
$ 10^5 $ электронов, им соответствует заряд $1,6 \cdot 10^{-14} $ Кл. Чтобы столь малое
количество заряда, создаваемое проходящей через камеру одной альфа-частицей, вызывало
измеряемое напряжение, емкость $ C $ должна быть мала. \\

При изменении давления в камере ионизационный ток меняется сначала линейно, а потом выходит
на насыщение. При небольших давлениях газа альфа-частицы передают часть энергии стенкам
камеры. По достижении давления $ P_0 $ все они заканчивают свой пробег внутри газа, и
дальнейшее возрастание тока прекращается. \\

В данной работе измерение пробега альфа-частицы проводится по величине тока ионизации в
сферической камере. Вакуумная установка содержит кран и манометр. Она позволяет изменять
давление в камере от атмосферного до 10 мм рт. ст. Величина тока ионизации измеряется
электрометром, состоящим из нескольких стандартных микросхем, по величине падения
напряжения на сопротивлении $ R $ = 100 МОм ($ C = 10^{-8} $ Фарад, так что $ RC $ = 1 с).
Значение измеряемого ионизационного тока (в пикоамперах) высвечивается на цифровом табло. \\

\subsection{Сцинтилляционный счетчик}

\pic
{0.3\linewidth}
{scheme2.jpg}
{Схема устройства cцинтилляционный счетчика}

\subsection{Счетчик Гейгера}

\pic
{0.3\linewidth}
{scheme3.jpg}
{Схема устройства счетчика Гейгера}

Для определения пробега альфа-частиц с помощью счетчика радиоактивный источник помещается
на дно стальной цилиндрической бомбы (рис. 3), в которой может перемещаться торцевой
счетчик Гейгера. Его чувствительный объем отделен от наружной среды тонким слюдяным окошком,
сквозь которое могут проходить альфа-частицы. Рабочее напряжение счетчика указано на
установке. \\

Импульсы, возникающие в счетчике, усиливаются и регистрируются пересчетной схемой. Путь
частиц в воздухе зависит от расстояния между источником и счетчиком. Перемещение счетчика
производится путем вращения гайки, находящейся на крышке бомбы. Расстояние между счетчиком
и препаратом измеряется по шкале, нанесенной на держатель счетчика. Счетчик не может быть
придвинут к препарату ближе чем на 10 мм, т. к. между источником и счетчиком установлен
коллиматор, изготовленный из плотно сжатых металлических трубок. Отверстия трубок
пропускают к счетчику только те альфа-частицы, которые вылетают из источника почти
перпендикулярно его поверхности.

\subsection{Вычисление энергии по длинне пробега}

Для энергий в диапазоне $ 4-9 $ МэВ выполняется эмперическое соотношение:

\formula{EmpiricalEnergy}
{E = 0.32 \cdot E^{\frac{3}{2}}}

\newpage
