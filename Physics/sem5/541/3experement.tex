
% Experement data + calcs goes here %

\section{Ход работы}

Сняли данные для всех трех установок.

\subsection{Ионизационная камера}

Представил результаты измерений на рисунке 4.

\pic
{0.8\linewidth}
{graph1.jpg}
{Зависимость $ I\left( P \right) $ в ионизационной камере}

По графику определил: \\

$ P\ruB{экстр} = \left( 570 \pm 10 \right) $ торр \\

Тогда рассчитаю пробег частиц: \\

$ R = \frac{288}{T} \cdot \frac{P}{760} \cdot \frac{10 - 0.5}{2} =
    \left(  3.42 \pm 0.08 \right) $ см \\

Где $ 0.5 $ и $ 10 $ - диаметры первого и второго электродов соответственно. \\

И соответствующую ему энергию частиц: \\

$ E = \left( \frac{R}{0.32} \right) ^ {\frac{2}{3}} = \left( 4.85 \pm 0.11 \right) $ МэВ \\

\newpage

\subsection{Сцинтилляционный счетчик}

Представил результаты измерений на рисунке 5.

\pic
{0.8\linewidth}
{graph2.jpg}
{Зависимость $ N(P) $ для сцинтилляционного счетчика}

По графику определил: \\

$ N''\left( P\ruB{ср} \right) = 0 \Rightarrow P\ruB{ср} = \left( 139 \pm 10 \right) $ торр \\

Тогда пробег: \\

$ R = \frac{288}{T} \cdot \frac{P}{760} \cdot 9 = \left( 1.61 \pm 0.12 \right) $ см \\

И энергия: \\

$ E = \left( 2.94 \pm 0.22 \right) $ МэВ \\

\subsection{Cчетчик Гейгера}

Представил результаты измерений на рисунке 6.

\pic
{0.8\linewidth}
{graph3.jpg}
{Зависимость $ N(x) $ для счетчика Гейгера}

\newpage

По графику определил: \\

$ N''\left( R\ruB{ср} \right) = 0 \Rightarrow R\ruB{ср} = \left( 18 \pm 1 \right) $ мм \\

Тогда энергия: \\

$ E = \left( 3.16 \pm 0.18 \right) $ МэВ \\


\section{Вывод}

В ходе данной работы энергия $ \alpha $-частиц была посчитана 3 способами. Полученные значения: \\

$ E_1 = \left( 4.85 \pm 0.11 \right) $ МэВ \\
$ E_2 = \left( 2.94 \pm 0.22 \right) $ МэВ \\
$ E_3 = \left( 3.16 \pm 0.18 \right) $ МэВ \\

Расхождение с реальным значением энергии частиц может быть обусловленно:

\begin{itemize}
    \item Конечными размерами пучков частиц, что приводит к угловой расходимости.
    \item Наличием слюдяной пленки на источнике излучения.
    \item Неточностью эмперической формулы \eqref{EmpiricalEnergy}
\end{itemize}