
% Experement data + calcs goes here %

\section{Ход работы}

\subsection{Подготовка установки}

Подготовили установку к работе.

\subsection{Градуировка спектрометра}

Проградуировали спектрометр по спектрам неона и ртути. Данные представлены в таблицах 1-2.

\begin{table}[h!]
    \begin{center}

        \tableLable{Спектр неона}
        \begin{tabular}{|l|c|c|c|c|c|c|c|c|c|c|c|c|c|}
        \hline
        N                   & 1    & 2    & 3    & 4    & 5    & 6    & 7    & 8    & 9    & 10   & 11   & 12   & 13   \\ \hline
        $ \Theta, \circ $   & 2574 & 2554 & 2484 & 2476 & 2440 & 2422 & 2418 & 2380 & 2372 & 2352 & 2342 & 2328 & 2308 \\ \hline
        $ \lambda, \, $ \AA & 7032 & 6929 & 6717 & 6678 & 6599 & 6533 & 6507 & 6402 & 6383 & 6334 & 6305 & 6267 & 6217 \\ \hline
        \end{tabular}




        \vspace{0.4cm}

        \begin{tabular}{|l|c|c|c|c|c|c|c|c|c|c|c|c|}
        \hline
        N                   & 14   & 15   & 16   & 17   & 18   & 19   & 20   & 21   & 22   & 23   & 24   & 25   \\ \hline
        $ \Theta, \circ $   & 2286 & 2276 & 2256 & 2246 & 2220 & 2200 & 2186 & 2156 & 2142 & 1884 & 1842 & 1834 \\ \hline
        $ \lambda, \, $ \AA & 6164 & 6143 & 6096 & 6074 & 6030 & 5976 & 5945 & 5882 & 5852 & 5401 & 5341 & 5331 \\ \hline
        \end{tabular}

    \end{center}
\end{table}

\begin{table}[h!]
    \begin{center}

        \tableLable{Спектр ртути}
        \begin{tabular}{|l|c|c|c|c|c|c|c|c|c|c|c|c|c|}
        \hline
        N                   & K1    & K2  & 1    & 2    & 3    & 4    & 5      & 6    \\ \hline
        $ \Theta, \circ $   & 2544 & 2318 & 2114 & 2104 & 1930 & 1488 & выброс & 838  \\ \hline
        $ \lambda, \, $ \AA & 6907 & 6234 & 5791 & 5770 & 5461 & 4916 & выброс & 4047 \\ \hline
        \end{tabular}

    \end{center}
\end{table}

\subsection{Спектр водорода}

Сняли спектр для водородной лампы. Данные представлены в таблице 3.

\begin{table}[h!]
    \begin{center}

        \tableLable{Спектр водорода}
        \begin{tabular}{|l|c|c|c|c|c|c|c|c|c|c|c|c|c|}
        \hline
        N                   & $ H_\alpha $ & $ H_\beta $ & $ H_\gamma $ & $ H_\delta $ \\ \hline
        $ \Theta, \circ $   & 2444         & 1456        & 814          & не видно     \\ \hline
        $ \lambda, \, $ \AA & 6600         & 4952        & 4020         & не видно     \\ \hline
        \end{tabular}

    \end{center}
\end{table}

\newpage

\subsection{График}

График со всеми данными градуировки и линиями спектра для водородной лампы представлены
на рисунке 3.

\pic{0.7\linewidth}{AllData.jpg}{Все данные градуировки и линии спектра водородной лампы}

\section{Обработка данных}

По полученным значениям спектра водорода посчитал постоянную Ридберга (формула
(\ref{LambdaHydrogen})): \\

$ R_\alpha = \left( 0.0109 \pm 0.0002 \right) \, \texttt{нм} ^{-1} $ \\
$ R_\beta  = \left( 0.0108 \pm 0.0002 \right) \, \texttt{нм} ^{-1} $ \\
$ R_\gamma = \left( 0.0109 \pm 0.0002 \right) \, \texttt{нм} ^{-1} $ \\

\section{Вывод}

В ходе данной работы был проградуирован спектрометр. И получены совпадающие с теорией
значения постоянной Ридберга. Точность измерений и градуировки можно улучшить путем
улучшения эргономики спектрометра (Проецировать изображение на поверхность?).
