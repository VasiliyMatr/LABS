
% Theory part goes here %

% for numerated formulas
\newcommand{\formula}[2]
{
    \begin{equation}\label{#1}
        #2
    \end{equation}
}

% for in-text math formulas
\newcommand{\mth}[1]
{
    \begin{math}
        #1
    \end{math}
}

% for rus letters in indexes
\newcommand{\ruB}[1]
{
    _{\text{#1}}
}

\newcounter{PicsCounter}
\setcounter{PicsCounter}{1}

\newcommand{\pic}[3]{
    \begin{center}
        \begin{minipage}[h!]{#1}
            \begin{center}

                \includegraphics[width = \textwidth]{#2}
                \textit{Рис \arabic{PicsCounter}. #3}

            \end{center}
        \end{minipage}
    \end{center}

    \stepcounter{PicsCounter}
}

\newcounter{TablesCounter}
\setcounter{TablesCounter}{1}

\newcommand{\tableLable}[1]{
    \textit{Таблица \arabic{TablesCounter}: #1}

    \stepcounter{TablesCounter}
}

\section{Теория}

\subsection{Спектр атомов водорода}

Объяснение структуры спектра излучения атомов требует решения задачи о движении электрона в эффективном поле атома.
Для атома водорода и водородоподобных (одноэлектронных) атомов определение энергетических уровней значительно
упрощается, так как квантово-механическая задача об относительном движении электрона (заряд $ -e $, масса $ m_e $)
и ядра (заряд $ Z_e $, масса $ M $) сводится к задаче о движении частицы с эффективной массой $ \mu = m_e M /(m_e+M) $
в кулоновском поле $ - Z \epsilon^2 / n $. Длины волн спектральных линий водородоподобного атома описываются формулой:

\formula{LambdaHydrogen}
{
    \frac{1}{\lambda_{m n}} = R Z^2 (\frac{1}{n^2}-\frac{1}{m^2})
}

Где $ m, \;n \in \mathbb{Z} $, а $ R $ -- постоянная Ридберга.
Эта формула позволяет по энергиям перехода судить о расположении энергетических уровней атома водорода. На рисунке 1
изображены энергетические уровни и соответствующие им переходы, определяющие спектр.

\pic
{0.6\linewidth}
{HydrogenSpectrum.jpg}
{Энергетические уровни атома водорода}

В данной работе изучается серия Бальмера, линии которой лежат в видимой области. Для серии Бальмера $ n = 2 $.
Величина $ m $ для первых четырех линий этой серии принимает значение 3, 4, 5, 6. Эти линии обозначаются символами
$ H_\alpha, \;H_\beta, \;H_\gamma, \;H_\delta $.

Оценим энергии основного и возбужденного состояний водородоподобного атома. Чтобы найти основное состояние квантовой
системы, надо минимизировать, с учетом соотношения неопределенностей, полную энергию. Потенциальная энергия электрона
равна кулоновской энергии электрона в поле ядра с зарядом $ Z e $. Так как электрон локализован в области размером
$ r $, то его импульс $ p \simeq \hbar / r $, и полная энергия определяется выражением:


\formula{HydrogenElectronEnergy}
{
    E = \dfrac{-Z e^2}{r}+\dfrac{\hbar^2}{2 m_e r^2}.
}

Приняв за нуль производную этого выражения, получим:

\formula{FirstBohrRadius}
{
    r\ruB{Б} = \frac{\hbar^2}{Z m_e e^2}.
}

Это значение радиуса первой орбиты для электрона в поле ядра с зарядом $ Z $ -- боровского радиуса.
Подставляя в (\ref{HydrogenElectronEnergy}) это значение, получим:

\formula{HydrogenElectronEnergyFinal}
{
    E = -R Z^2
}

При этом:

\formula{RydbergConstant}
{
    R = \frac{m_e e^4}{2 \hbar^2}.
}

Для возбуждённых состояний значения энергий можно найти аналогично, приняв во внимание, что $ p \simeq n \hbar / r $
из условия, что на длине орбиты укладывается целое число волн де Бройля. Отсюда энергия $ n $-го уровня равна:

\formula{HydrogenElectronEnergyN}
{
    E = \frac{-R Z^2}{n^2}
}

\subsection{Спектр молекул йода}

Спектр молекулярного йода представлен на рисунке 1.

\pic
{0.6\linewidth}
{IodineSpectrum.jpg}
{Спектральная картина йода}

Массы ядер атомов велики по сравнению с массой электрона. Благодаря такой разнице в массах, скорости движения ядер
в молекуле малы по сравнению со скоростями электронов. Это даёт возможность рассматривать электронное движение при
неподвижных ядрах, расположенных на определенных расстояниях друг от друга. Определяя уровни энергии такой системы,
мы найдем электронные термы молекул. Любой атом в молекуле находится в электрическом поле остальных ее атомов.
Оно вызывает расщепление электронных уровней атомов в молекуле. Следует отметить, что при соединении атомов в
молекулу заполненные оболочки атомов мало меняются. Существенно может измениться распределение электронной плотности
в не до конца заполненных оболочках. \\

Для расчёта спектра поглощения йода необходимо учесть энергии колебательного и вращательного движения молекул.
Видимый спектр состоит из 0-й и 1-й серий Деландра. 2-я серия в 10 раз менее интенсивная, чем 0-я, и поэтому ей
пренебрегаем.

Энергетическое положение линий поглощения описывается выражением:
\formula{IodineEnergyLines}
{
    h \nu_{0 n_2} = (E_2 - E_1 )+ h \nu_2 \left(n_2+\dfrac{1}{2}\right) - \dfrac{1}{2}h \nu_1.
}
