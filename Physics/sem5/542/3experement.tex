
% Experement data + calcs goes here %

\section{Ход работы}

Откачали воздух из полости спектрометра, включили вакуумметр, ПЭВМ, формирователь
импульсов и питание магнитной линзы. Уменьшили ток через магнитную линзу до нуля. \\

Провели измерения $ \beta $-спектра, изменяя ток в магнитной линзе, при каждом
значении тока будем сняли число попаданий частиц в детектор за $ 100 $ секунд.
Значения сразу будут приведены в виде $ N = \frac{N'}{t} $ -- число частиц в
единицу времени. Результаты представлены на рисунке 4. \\

Так же учли фон: \\
$ N\ruB{Ф} = 0.750 \, \texttt{с}^{-1} $

\pic{0.8\linewidth}{Graph1}{Эксперементальный $ \beta $-спектр}

Нашел постоянную установки: \\
$ K = \left(235 \pm 6\right) \, \frac{keV}{A} \cdot c^{-1} $ \\

Зная конверсионный пик и соответствующие ему импульс $ p_c = 1013.5 $ кэВ/с и
энергию $ T = 634 $ кэВ, откалибровал шкалу токов в шкалу импульсов и энергий. \\

Из формулы (\ref{dNdE}) следует:

\formula{FermiCurie}{
    \dfrac{\sqrt{N}}{p^{3/2}} \propto T_{max} - T
}

Отложив по оси $ y $ величину $ \dfrac{\sqrt{N}}{p^{3/2}} = f $, а по $ x $ --- кинетическую
энергию, построил график, называемый графиком Ферми-Кюри (рис. 5), и определил
по нему $ T_{max} $.

\pic{0.8\linewidth}{Graph2.jpg}{График Ферми-Кюри}

Получил: \\
$ T_{max} = \left( 688 \pm 10 \right) \, $ кэВ

\section{Вывод}

Таким образом, в ходе данной работы был изучен спектр $ \beta $-распада $ ^{136} Cs $. Так
же были исследованы различные свойства данного спектра и установки: конверсионный пик,
параметр установки и максимальная кинетическая энергия в данном распаде. 
