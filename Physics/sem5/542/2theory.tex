
% Theory part goes here %

% for numerated formulas
\newcommand{\formula}[2]
{
    \begin{equation}\label{#1}
        #2
    \end{equation}
}

% for in-text math formulas
\newcommand{\mth}[1]
{
    \begin{math}
        #1
    \end{math}
}

% for rus letters in indexes
\newcommand{\ruB}[1]
{
    _{\text{#1}}
}

\newcounter{PicsCounter}
\setcounter{PicsCounter}{1}

\newcommand{\pic}[3]{
    \begin{center}
        \begin{minipage}[h!]{#1}
            \begin{center}

                \includegraphics[width = \textwidth]{#2}
                \textit{Рис \arabic{PicsCounter}. #3}

            \end{center}
        \end{minipage}
    \end{center}

    \stepcounter{PicsCounter}
}

\newcounter{TablesCounter}
\setcounter{TablesCounter}{1}

\newcommand{\tableLable}[1]{
    \textit{Таблица \arabic{TablesCounter}: #1}

    \stepcounter{TablesCounter}
}

\section{Теория}

Бета-распадом называется самопроизвольное превращение ядер, при котором их
массовое число не изменяется, а заряд увеличивается или уменьшается на единицу.
Бета-активные ядра встречаются во всей области значений массового числа $ A $,
начиная от единицы (свободный нейтрон) и заканчивая самыми тяжелыми ядрами. Период
полураспада $ \beta $-активных ядер изменяется от ничтожных долей секунды до
$ 10^{18} $ лет. Выделяющаяся при единичном акте $\beta$-распада энергия
варьируется от $ 18 $ кэВ до $ 13.4 $ МэВ. \\

В данной работе мы будем иметь дело с электронным распадом:
\formula{ElectronicDecay}{
    ^A_ZX \rightarrow ^{\; \; \; \; \:A}_{Z + 1}X + e^{ -} + \widetilde{\nu}
}

При данном распаде кроме электрона испускается антинейтрино. Освобождающаяся при
$ \beta $-распаде энергия делится между электроном, антинейтрино и дочерним ядром,
однако доля энергии, передаваемой ядру, мала по сравнению с энергией,
уносимой электроном и антинейтрино. Практически можно считать, что эти две
частицы делят между собой всю освобождающуюся энергию. Поэтому электроны могут
иметь любое значение энергии  от нулевой до некоторой максимальной, которая
равна энергии, освобождающейся при $ \beta $-распаде, являющейся важной физической
величиной. \\

Вероятность $ dw $ того, что при распаде электрон вылетит с импульсом в
интервале $ d^3p $, а антинейтрино с импульсом в интервале $ d^3k $, пропорциональна
произведению этих дифференциалов. Но мы должны еще учесть закон сохранения
энергии, согласно которому импульсы $ p $ и $ k $ электрона и антинейтрино
связаны соотношением:

\formula{EnergyConservation}{
    E_e - E - ck = 0
}

где $ E_e $ --- максимальная энергия электрона, кинетическая энергия электрона
$ E $ связана с его импульсом обычным релятивистским соотношением:

\formula{ElectronEnergy}{
    E = c\sqrt{p^2 + m^2c^2} - mc^2
}

Через $ ck $ обозначена энергия антинейтрино с импульсом $ k $. Условие можно
учесть введением в выражение для $ dw $ $ \delta $-функции:

\formula{DeltaFunction}{
    \delta(E_e - E - ck)
}

Таким образом, вероятность $ dw $ может быть записана в виде:

\formula{Probability}{
    dw = D \delta (E_e - E - ck)d^3 p d^3 k =
    D \delta (E_e - E - ck)p^2dpk^2dkd{\Omega}_ed{\Omega}_{\widetilde{\nu}},
}

Где $ D $ --- некоторый коэффициент пропорциональности, $ d\Omega_e $,
$ d\Omega_{\widetilde{\nu}} $ --- элементы телесных углов направлений вылета
электрона и нейтрино. Вероятность $ dw $ непосредственно связана с
$ \beta $-спектром, поскольку для большого числа $ N_0 $ распадов число $ dN $
распадов с вылетом электрона и антинейтрино с импульсом соответственно от $ p $
до $ p + dp $ и от $ k $ до $ k + dk $ определяется соотношением:

\formula{dNdw}{
    dN = N_0 dw
}

\newpage

Коэффициент $ D $ в формуле (\ref{Probability}) можно считать для рассматриваемых нами
так называемых разрешённых фермиевских типов распадов с хорошей точностью
константой (разрешёнными называются такие переходы, при которых не изменяются ни
момент, ни чётность состояния ядра). В этом случае величину $ dw $ из
(\ref{dNdw}) можно проинтегрировать по всем углам и по абсолютному значению
импульса нейтрино. \\

После умножения на полное число распадов $ N $ проинтегрированное выражение
приобретает смысл числа электронов $ dN $, вылетающих из ядра с импульсом,
абсолютная величина которого лежит между $ p $ и $ p + dp $:

\formula{dN}{
    dN = \frac{16\pi^2 N_0}{c^2}Dp^2(E_e - E)^2dp
}

Чтобы получить распределение электронов по энергиям, надо в (\ref{dN}) перейти
от $ dp $ к $ dE $:

\formula{dE}{
    dE = \frac{c^2p}{E + mc^2}dp
}

После чего выражающая форму $ \beta $-спектра величина $ N(E) = \frac{dN}{dE} $
приобретает вид:

\formula{dNdEex}{
    \frac{dN}{dE} =
    N_0Bcp(E + mc^2)(E_e - E)^2 = N_0B\sqrt{E(E + 2mc^2)}(E_e - E)^2(E + mc^2)
}

Где $ B = \frac{16\pi^2}{c^4}D $. В нерелятивистском приближении, которое и имеет
место в нашем случае, выражение (\ref{dNdEex}) упрощается, и мы имеем:

\formula{dNdE}{
    \frac{dN}{dE} \approx \sqrt{E}(E_e - E)^2
}

Выражение (\ref{dNdE}) приводит к спектру, имеющему вид широкого колокола (рис 1).
Кривая плавно отходит от нуля и столь же плавно, по параболе, касается оси абсцисс в
области максимальной энергии электронов $ E_e $.

\pic{0.3\linewidth}{BetaShape.jpg}{Форма спектра $ \beta $-частиц при разрешённых переходах.}

Дочерние ядра, возникающие в результате $ \beta $-распада, нередко оказываются
возбуждёнными. Возбуждённые ядра отдают свою энергию либо излучая $ \gamma $-квант
(энергия которого равна разности энергий начального и конечного уровней), либо
передавая избыток энергии одному из электронов с внутренних оболочек атома.
Излучаемые в таком процессе электроны имеют строго определённую энергию и
называются конверсионными. \\

Конверсия чаще всего происходит на оболочках $ K $ или $ L $. На спектре, представленном на
рис. 1, видна монохроматическая линия, вызванная электронами конверсии. Ширина этой линии
в нашем случае является чисто аппаратурной, по ней можно оценить разрешающую силу спектрометра.

\newpage

\section{Эксперементальная установка}

Для определения энергии $ \beta $-частиц в работе используется магнитный
спектрометр, схема которого показана на рисунке 1. Электроны испускаются радиоактивным
источником и попадают в магнитное поле катушки, ось которой параллельна $ OZ $.

\pic{0.6\linewidth}{Scheme1.jpg}{Схема $ \beta $-спектрометра с короткой магнитной линзой.}

Траектории электронов сходятся в одной точке --- фокусе, где и установлен
сцинтилляционный счетчик, сигналы которого усиливаются фотоумножителем и
регистрируются пересчётным прибором. Фокусное расстояние $ f $ магнитной линзы
связано с током в катушке $ I $ и импульсом $ p_e $ регистрируемых частиц следующим
образом:

\formula{fProportionality}{
    \frac{1}{f} \propto \frac{I^2}{p_e^2}
}

При неизменной геометрии установки, увеличивая и уменьшая силу тока, можно
фокусировать электроны разных импульсов, причём:

\formula{pek}{
    p_e = kI
}

где $ k $ --- коэффициент пропорциональности, являющийся параметром установки.

Блок-схема установки изображена на рисунке 3.

\pic{0.6\linewidth}{Scheme2}{Блок-схема установки для изучения $\beta$-спектра.}

Радиоактивный источник $ ^{137} $Cs помещён внутрь откачанной трубы.
Электроны, сфокусированные магнитной линзой, попадают в счётчик.
В газоразрядном счётчике они инициируют газовый разряд и тем самым приводят к
появлению электрических импульсов на его электродах, которые затем регистрируются
пересчётным прибором. В результате попадания электронов в сцинтиллятор на выходе
фотоумножителя появляются электрические импульсы, которые заносятся в память
персонального компьютера и выводятся на экран монитора. Давление в спектрометре
поддерживается на уровне около $ 0.1 $ Тор и измеряется термопарным вакууметром.
Лучший вакуум в приборе не нужен, поскольку уже при этом давлении потери энергии
электронов малы и их рассеяние незначительно. Откачка осуществляется форвакуумным
насосом. Магнитная линза питается постоянным током от выпрямителя. Ток можно
повышать до $ 6 $ А, он измеряется цифровым прибором. Высокое напряжение на ФЭУ
или газоразрядный счётчик подаётся от стабилизированного выпрямителя.
