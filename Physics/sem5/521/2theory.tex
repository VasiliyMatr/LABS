
% Theory part goes here %

% for numerated formulas
\newcommand{\formula}[2]
{
    \begin{equation}\label{#1}
        #2
    \end{equation}
}

% for in-text math formulas
\newcommand{\mth}[1]
{
    \begin{math}
        #1
    \end{math}
}

% for rus letters in indexes
\newcommand{\ruB}[1]
{
    _{\text{#1}}
}

\newcounter{PicsCounter}
\setcounter{PicsCounter}{1}

\newcommand{\pic}[3]{
    \begin{center}
        \begin{minipage}[h!]{#1}
            \begin{center}

                \includegraphics[width = \textwidth]{#2}
                \textit{Рис \arabic{PicsCounter}. #3}

            \end{center}
        \end{minipage}
    \end{center}

    \stepcounter{PicsCounter}
}

\newcounter{TablesCounter}
\setcounter{TablesCounter}{1}

\newcommand{\tableLable}[1]{
    \textit{Таблица \arabic{TablesCounter}: #1}

    \stepcounter{TablesCounter}
}

\section{Теория}

\subsection{Принципиальная схема опыта Франка-Герца}

Принципиальная схема представлена на рисунке 1.

\pic
{0.8\linewidth}
{scheme1.jpg}
{Принципиальная схема опыта Франка-Герца}

\subsection{Теоретические выкладки}

Одним из простых опытов, подтверждающих существование дискретных уровней энергии атомов,
является эксперимент, известный под названием опыта Франка-Герца. \\

Разреженный одноатомный газ (в нашем случае -- гелий) заполняет трёхэлектродную лампу.
Электроны, испускаемые разогретым катодом, ускоряются в постоянном электрическом поле,
созданным между катодом и сетчатым анодом лампы. Передвигаясь от катода к аноду, электроны
сталкиваются с атомами гелия. \\

При столкновении электрона с атомом газа возможны следующие варианты:

\begin{enumerate}
    \item У электрона достаточно энергии $ \longrightarrow $ происходит возбуждение
    атомного электрона.
    \item У электрона не достаточно энергии $ \longrightarrow $ электрон не меняет
    свою энергию.
\end{enumerate}

Таким образом, на ВАХ трехэлектродной лампы (см рис. 1) будет наблюдаться ряд
последовательных максимумов и минимумов. При этом, максимумы и минимумы будут
отстоять друг от друга на равные расстояния $ \Delta V $. А $ \Delta V $ будет
соответствовать энергии первого возбужденного состояния.
