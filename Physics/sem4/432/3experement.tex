
% Experement data + calcs goes here %

\section{Схема установки}

\begin{enumerate}
    \item Определение скорости УЗ волны по дифф картине:

    \pic{0.95\linewidth}{scheme1.png}{Схема установки 1}
    \vspace{3cm}

    \item Определение скорости УЗ волны методом темного поля:
    
    \pic{0.95\linewidth}{scheme2.png}{Схема установки 2}
\end{enumerate}

\newpage

\section{Ход работы}

\subsection{Параметры установки}

Параметры установки:
\begin{itemize}
    \item фокусное расстояние объектива F = 30 см
    \item Цена деления микрометрического винта на микроскопе d = 4 мкм
    \item Частота проявления дифракционной решетки $\nu_0$ = 1.17 МГц
\end{itemize}

\subsection{Эксперементальные данные}

Положения дифф максимумов $X_m$ при $\nu$ = 1.17 МГц представлены в таблице 1.
\begin{table}[h!]
\begin{center}
    \caption{}
    \begin{tabular}{|c|c|c|c|c|c|c|c|c|c|}
    \hline
    m          & -4   & -3   & -2   & -1   & 0 & 1   & 2   & 3   & 4   \\ \hline
    $X_m$, мкм & -604 & -452 & -304 & -152 & 0 & 148 & 292 & 440 & 588 \\ \hline
    \end{tabular}
\end{center}
\end{table}

Положения дифф максимумов $X_m$ при $\nu$ = 1.28 МГц представлены в таблице 2.
\begin{table}[h!]
\begin{center}
    \caption{}
    \begin{tabular}{|c|c|c|c|c|c|c|c|}
    \hline
    m          & -3   & -2   & -1   & 0 & 1   & 2   & 3   \\ \hline
    $X_m$, мкм & -496 & -328 & -160 & 0 & 164 & 348 & 516 \\ \hline
    \end{tabular}
\end{center}
\end{table}

Положения дифф максимумов $X_m$ при $\nu$ = 1.40 МГц представлены в таблице 3.
\begin{table}[h!]
\begin{center}
    \caption{}
    \begin{tabular}{|c|c|c|c|c|c|c|c|}
    \hline
    m          & -3   & -2   & -1   & 0 & 1   & 2   & 3   \\ \hline
    $X_m$, мкм & -544 & -360 & -180 & 0 & 184 & 368 & 548 \\ \hline
    \end{tabular}
\end{center}
\end{table}

Положения дифф максимумов $X_m$ при $\nu$ = 1.55 МГц представлены в таблице 4.
\begin{table}[h!]
\begin{center}
    \caption{}
    \begin{tabular}{|c|c|c|c|c|c|c|c|}
    \hline
    m          & -3   & -2   & -1   & 0 & 1   & 2   & 3   \\ \hline
    $X_m$, мкм & -596 & -392 & -196 & 0 & 196 & 404 & 600 \\ \hline
    \end{tabular}
\end{center}
\end{table}

\newpage

\subsection{Графики}

Графики для зависимостей $X_m(m)$ представлены на рисунках 3 - 6.

\pic{0.6\linewidth}{graph1.png}{График для $\nu$ = 1.17 МГц}
\pic{0.6\linewidth}{graph2.png}{График для $\nu$ = 1.28 МГц}
\pic{0.6\linewidth}{graph3.png}{График для $\nu$ = 1.40 МГц}
\pic{0.6\linewidth}{graph4.png}{График для $\nu$ = 1.55 МГц}

\subsection{Построение зависимости $\Lambda(1/\nu)$}

Коэффициенты пропорциональности для зависимостей $X_m(m)$ и значения $\Lambda$ для
различных частот представлены в таблице 5.

\begin{table}[h!]
    \begin{center}
        \caption{}
        \begin{tabular}{|c|c|c|}
        \hline
        $\nu$, МГц & $\tau$, мкм & $\Lambda$, мкм \\ \hline
        1.17 & 149 & 1289 \\ \hline
        1.28 & 168 & 1141 \\ \hline
        1.40 & 182 & 1055 \\ \hline
        1.55 & 199 &  965 \\ \hline
        \end{tabular}
    \end{center}
\end{table}

График для зависимости $\Lambda(1/\nu)$ представлен на рисунке 7.

\pic{0.6\linewidth}{graph5.png}{График $\Lambda(1/\nu)$}

\noindentИз к-та наклона графика получаем скорость распространения УЗ в воде:
$V\ruB{УЗ} = (1498\pm50)$ м/с.

\newpage

\section{Вывод}

В работе изучена дифракция света на аккустической решетке. Построены графики для
зависимостей $X_m(m)$ при четырех разных частотах. Так же получена зависимость
$\Lambda(1/\nu)$. Рассчитаны длинны волн УЗ и скорость его распространения в воде.

