
% Theory part goes here %

% for numerated formulas
\newcommand{\formula}[3]
{
    \noindent#1\\[0.1cm]
    \begin{equation}\label{#2}
        #3
    \end{equation}
}

% for in-text math formulas
\newcommand{\mth}[1]
{
    \begin{math}
        #1
    \end{math}
}

% for rus letters in indexes
\newcommand{\ruB}[1]
{
    _{\text{#1}}
}

\newcounter{PicsCounter}
\setcounter{PicsCounter}{1}

\newcommand{\pic}[3]{
    \begin{center}
    \begin{minipage}[h!]{#1}
    \begin{center}

    \includegraphics[width = \textwidth]{picks/#2}
    \textit{Рис \arabic{PicsCounter}. #3}

    \end{center}
    \end{minipage}
    \end{center}

    \stepcounter{PicsCounter}
}

\section{Теория}

При прохождении ультразвуковой волны через жидкость в ней возникают периодические
неоднородности коэффициента преломления, создается фазовая решетка, которую мы считаем
неподвижной ввиду малости скорости звука относительно скорости света.

\formula
{Показатель n преломления изменятеся по закону:}
{NLaw}
{n = n_0\left(1 + m\cos{\Omega x}\right),}

где $\Omega = 2\pi/\Lambda$ - волновое число УЗ волны, m - глубина модуляции n (m << 1).

\formula
{Фаза $\Phi$ колебаний световой волны на задней стенке кюветы:}
{Phase}
{\Phi = \Phi_0\left(1 + m\cos{\Omega x}\right)}

После прохождения через кювету световое поле есть совокупность плоских волн,
распространяющихся под углами $\Theta$, соответствующими максимумам
дифракции Фраунгофера (см рис. 1):

\formula
{}
{Theta}
{\Lambda\sin{\Theta_m} = m\lambda}

\pic{0.6\linewidth}{diffract.png}{Дифракционная картина}

\vspace{3cm}

\formula
{Длинна УЗ волны высчитывается по формуле}
{Lambda}
{\Lambda=m\frac{\lambda F}{l_m}}

\newpage
