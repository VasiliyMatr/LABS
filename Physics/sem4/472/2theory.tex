
% Theory part goes here %

% for numerated formulas
\newcommand{\formula}[2]
{
    \begin{equation}\label{#1}
        #2
    \end{equation}
}

% for in-text math formulas
\newcommand{\mth}[1]
{
    \begin{math}
        #1
    \end{math}
}

% for rus letters in indexes
\newcommand{\ruB}[1]
{
    _{\text{#1}}
}

\newcounter{PicsCounter}
\setcounter{PicsCounter}{1}

\newcommand{\pic}[3]{
    \begin{minipage}[h!]{#1}
    \begin{center}

    \includegraphics[width = \textwidth]{#2}
    \textit{Рис \arabic{PicsCounter}. #3}

    \end{center}
    \end{minipage}
    \stepcounter{PicsCounter}
}

\newcounter{TablesCounter}
\setcounter{TablesCounter}{1}

\newcommand{\tableLable}[1]{
    \textit{Таблица \arabic{TablesCounter}: #1}

    \stepcounter{TablesCounter}
}

\section{Теоретическая часть}

\subsection{Схема установки}

\begin{center}
    \pic{0.4\linewidth}{interfeer.jpg}{Схема для наблюдения интерфереционной картины}
    \pic{0.4\linewidth}{polar.jpg}{Схема для наблюдения влияния на поляризацию}
\end{center}

Эффект Поккельса -- изменение показателя преломления света в кристалле под действием
электрического поля. \\

В данной работе используется кристалл ниобата лития $ LiNbO_3 $ с центральноосевой
симметрией вдоль оси $ Z $. Для световой волны с $ E $, ортогональным $ Z $, показатель
преломления будет $ n_0 $. А для волны с $ E $, сонаправленным с $ Z $ -- $ n_e $.
В случае, когда луч света идёт под углом $ \theta $ к оси, есть два значение показателя
преломления $ n_1 $ и $ n_2 $: $ n_1 = n_0 $ для волны с $ E $, перпендикулярным плоскости
$ \left( k, Z \right) $ (обыкновенная волна), и $n_2$ -- для волны с $ E $ в этой плоскости
(необыкновенная волна). В последнем случае: \\

\formula{n2}{
    \frac{1}{n_2^2}=\frac{\cos^2 \theta}{n_0^2}+\frac{\sin^2 \theta}{n_e^2}
}

Если перед кристаллом, помещённым между поляроидами, расположить линзу или матовую
пластинку, то на экране за поляроидом мы увидим тёмные концентрические окружности --
результат интерфернции обыкновенной и необыкновенной волн. При повороте выходного поляроида
на $ 90^\circ $ картина меняется с позитива на негатив (на месте светлых пятен тёмные и
наоборот). В случаи, когда разрешённое направление анализатора перпендикулярно поляризации
лазерного излучения, радиус тёмного кольца с номером $ m $ равен: \\

\formula{rm}{
    r_m^2 = \frac{\lambda}{l} \frac{(n_0L)^2}{n_0 - n_e}m ,
} \\

где $ L $ -- расстояние от центра кристалла до экрана, $ l $ -- длина кристалла. \\

Теперь поместим кристалл в постоянное электрическое поле $ E\ruB{эл} $, направленное вдоль
оси $ X $, перпендикулярной $ Z $. Показатель преломления для луча, распространяющегося
вдоль $ Z $, всегда $ n_0 $. В плоскости $ \left( X, Y \right) $ возникают два главных
направления под углами $ 45^\circ $ к $ X $ и $ Y $ с показателями преломления
$ n_0 - \Delta n $ и $ n_0 + \Delta n $ (быстрая и медленная ось), причём
$ \Delta n = A E\ruB{эл} $. Для поляризованного вертикально света и анализатора,
пропускающего горизонтальную поляризацию, на выходе интенсивность будет иметь вид: \\

\formula{Iout}{
    I\ruB{вых} = I_0 \sin^2 \left( \frac{\pi}{2} \frac{U}{U_{\lambda/2}} \right) ,
} \\

где $ U_{\lambda/2} = \frac{\lambda}{4A} \frac{d}{l} $ -- полуволновое напряжение, $ d $ --
поперечный размер кристалла. При напряжении $ U = E\ruB{эл}d $, равном полуволновому сдвигу
фаз между двумя волнами, интенсивность света на выходе максимальна. \\

Свет лазера, проходя через пластину, рассеивается и падает на двоякопреломляющий кристалл.
На экране за поляроидом видна интерференционная картина. Убрав рассеивающую пластину и
подавая на кристалл постоянное напряжение, можно величиной напряжения влиять на поляризацию
луча, вышедшего из кристалла. Заменив экран фотодиодом и подав на кристалл переменное напряжение,
можно исследовать поляризацию с помощью осциллографа.

