
% Experement data + calcs goes here %

\section{Ход работы}

\subsection{Настройка системы}

Произвел юстировку системы. \\

Параметры системы: \\
$ \lambda = 0,63 \, \texttt{мкм} $ \\
$ n_0 = 2,29 $ \\
размеры кристалла: $ 3 X 3 X 26 \, \texttt{мм} $
$ L = \left( 91\pm 3 \right) \, \texttt{см} $ - расстояние от середины кристалла до экрана.

\subsection{Наблюдение темных колец}

Собрал установку, изображенную на Рисунке 1. \\

Собранные данные представлены в Таблице 1.

\begin{table}[h!]
    \begin{center}
        \tableLable{Радиусы темных колец}
        \begin{tabular}{|l|l|l|l|l|l|l|l|}
        \hline
        $ m $     & 1   & 2   & 3   & 4   & 5   & 6   & 7   \\ \hline
        $ r, cm $ & 3,1 & 4,4 & 5,5 & 6,3 & 7,1 & 7,7 & 8,3 \\ \hline
        \end{tabular}
    \end{center}
\end{table}

\begin{center}
\pic{0.7\linewidth}{graph.jpg}{Зависимость $ r^2(m) $}
\end{center}

По МНК определяю угловой коэффициент зависимости $ r^2(m) $. \\
$ k = \left( 9,84 \pm 0,11 \right) cm^2 $. \\

Тогда получаю по формуле (2): \\
$ n_0 - n_e = 0,105 \pm 0,010 $.

\subsection{Изменение характера поляризации во внешнем поле}

Собрал схему, изображенную на Рисунке 2. \\

Собранные данные: \\

1 min: 0 кВ \\
1 max: 0,24 кВ \\
2 min: 0,47 кВ \\
2 max: 0,75 кВ \\

Точно такие же значения получены для вертикального направления.

Тогда: $ U_{\lambda/2} = \left( 0,24 \pm 0,02 \right) \, \texttt{кВ}$

\subsection{Фигуры Лиссажу}

Добавил вместо экрана фотодиод. Пронаблюдал фигуры Лиссажу и записал для данных фигур
подаваемое напряжение. \\

$ U_1 = 0,18 \, \texttt{кВ} $ \\
$ U_2 = 0,39 \, \texttt{кВ} $ \\
$ U_1 = 0,62 \, \texttt{кВ} $ \\

Тогда: $ U_{\lambda/2} = \left( 0,22 \pm 0,02 \right) \, \texttt{кВ}$

\pic{0,4\linewidth}{Lissajous1.jpg}{Первая фигура Лиссажу}
\pic{0,4\linewidth}{Lissajous2.jpg}{Вторая фигура Лиссажу}
\pic{0,4\linewidth}{Lissajous3.jpg}{Третья фигура Лиссажу}

\section{Вывод}

Таким образом, в ходе данной лабораторной работы, была исследована интерференция рассеянного
света, прошедшего кристалл. Также пронаблюдал изменение характера поляризации света при
наложении на кристалл электрического поля.
