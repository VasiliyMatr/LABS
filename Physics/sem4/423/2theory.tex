
% Theory part goes here %

% for numerated formulas
\newcommand{\formula}[2]
{
    \begin{equation}\label{#1}
        #2
    \end{equation}
}


% for in-text math formulas
\newcommand{\mth}[1]
{
    \begin{math}
        #1
    \end{math}
}

% for rus letters in indexes
\newcommand{\ruB}[1]
{
    _{\text{#1}}
}

\newcounter{PicsCounter}
\setcounter{PicsCounter}{1}

\newcommand{\pic}[3]{
    \begin{center}
    \begin{minipage}[h!]{#1}
    \begin{center}

    \includegraphics[width = \textwidth]{#2}
    \textit{Рис \arabic{PicsCounter}. #3}

    \end{center}
    \end{minipage}
    \end{center}

    \stepcounter{PicsCounter}
}

\section{Схема установки}

Интерферометер Релея --- прибор для измерения разности показателей преломления --- основан
на явлении дифракции света на двух параллельных щелях. Схема установки представлена на
рисунке 1 в вертикальной и горизонтальных проекциях.

\pic{0.7\linewidth}{scheme.jpg}{Схема установки: а) вид сверху; б) вид сбоку}

\section{Теория}

Формула для рассчета изменения n:
\formula
{ deltaN }
{ \Delta n = \frac{\Delta}{l} = m \frac{\lambda}{l} }

\noindentПоказатель преломления исследуемого газа определяется путем сравнения с воздухом:
\formula
{ nAir }
{ n = n\ruB{возд} + \frac{\Delta}{l} }

\noindentЗависимость показателя преломления газа от давления и температуры:
\formula
{ nFromPnT }
{ n = \sqrt{\epsilon} = \sqrt{1 + 4\pi N \alpha} \approx 1 + 2\pi N \alpha }

\noindentС учетом соотношения $ P = NkT $, получим:
\formula
{ nWithRelation }
{ n - 1 = \frac{2\pi\alpha}{kT}P }

\noindentТогда при постоянной температуре:
\formula
{ nWithConstT }
{ \Delta n = \frac{2\pi\alpha}{kT}\Delta P }

\noindentТогда связь для $ n $ и $ n_0 $:
\formula
{ nFinal }
{ \frac{n_0 - 1}{n - 1} = \frac{TP_0}{T_0P} }

