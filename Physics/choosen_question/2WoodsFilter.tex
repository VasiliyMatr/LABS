
% for numerated formulas
\newcommand{\formula}[2]
{
    \begin{equation}\label{#1}
        #2
    \end{equation}
}

% for in-text math formulas
\newcommand{\mth}[1]
{
    \begin{math}
        #1
    \end{math}
}

% for rus letters in indexes
\newcommand{\ruB}[1]
{
    _{\text{#1}}
}

\newcounter{PicsCounter}
\setcounter{PicsCounter}{1}

\newcommand{\pic}[3]{
    \begin{center}
    \begin{minipage}[h!]{#1}
    \begin{center}

    \includegraphics[width = \textwidth]{#2}
    \textit{Рис \arabic{PicsCounter}. #3}

    \end{center}
    \end{minipage}
    \end{center}

    \stepcounter{PicsCounter}
}

\newcounter{TablesCounter}
\setcounter{TablesCounter}{1}

\newcommand{\tableLable}[1]{
    \textit{Рис \arabic{TablesCounter}: #1}

    \stepcounter{TablesCounter}
}

\section{Фильтр Вуда}

\subsection{Интерференционно-поляризационный фильр Вуда}

Интерференционно-поляризационный фильр Вуда состоит из пластинки одноосного кристалла C,
вырезанной параллельно оптической оси, помещенной между двумя поляризаторами A и B. Схема
представлена на Рисунке 1. Оси поляризаторов обычно устанавливают параллельно, а ось
кристаллической пластинки составляет с ними угол в $ 45^\circ $

\pic{0.5\linewidth}{WoodsFilter.jpg}{Интерференционно-поляризационный фильр Вуда}

Поляризованный пучок света в пластинке C расщепляется на два одинаково направленных, равных
по интенсивности и поляризованных во взаимно перпендикулярных направлениях пучка света.
Эти пучки распространяются в кристалле с разными скоростями: $ v_o = \frac{c}{n_o} $ и
$ v_e = \frac{c}{n_e} $, где $ n_o $ и $ n_e $ -- показатели преломления для обыкновенного
и необыкновенного лучей. Результат интерференции этих лучей при выходе из пластинки
определяется приобретенной ими разностью хода -- см. Рисунок 2. Если разность хода
$ \Delta = l\left( n_e - n_o \right) $ равна целому числу длин волн, то на выходе получается
плоско поляризованный свет с первоначальной ориентацией плоскости поляризации. Такой свет
беспрепятственно проходит через второй поляризатор. Если разность хода равна нечетному числу
длин полуволн, то плоскость колебаний развернется на $ 90^\circ $. В этом случае свет будет
полностью задержан вторым поляризатором. \\ [0.3cm]

\pic{0.7\linewidth}{polarization.jpg}{Поляризация после прохождения пластинки C}

\newpage

Таким образом, нетрудно показать, что пропускание системы будет зависеть от длины волны
следующим образом:

\formula{WoodsFilterTransmission}{
    T = \cos^2\pi \frac{\left( n_e - n_o \right) l}{\lambda}
} \\

Разумеется, данная формула не учитывает потерь на отражение и поглощение света элементами
фильтра, а так же потерь в первом поляризаторе. Однако, данная формула позволяет понять,
что пропускание интерференционно-поляризационного фильтра имеет ряд максимумов (T = 1)
для длин волн:

\formula{WoodsFilterTransmissionMax}{
    \lambda_{max} = \frac{1}{k} l \left( n_e - n_o \right)
} \\

И минимумов:

\formula{WoodsFilterTransmissionMin}{
    \lambda_{min} = \frac{2}{2k + 1} l \left( n_e - n_o \right)
} \\

Поэтому такой фильтр может служить лишь для гашения отдельных спектральных линий и
пропускания других. Обычно его используют для выделения одной из двух близко расположенных
линий. \\

Спектральный интервал между соседними максимумом и минимумом пропускания фильтра,
определяющий полуширину полосы пропускания легко найти из формул
(\ref{WoodsFilterTransmissionMax}) и (\ref{WoodsFilterTransmissionMin}):

\formula{WoodsFilterDeltaInter}{
    \Delta\lambda = \lambda_{max} - \lambda{min} = \frac{\lambda_{max}}{2k + 1}
} \\

Пренебрегая единицей в сравнении с $ 2k $, получаем формулу:

\formula{WoodsFilterDelta}{
    \Delta\lambda \approx \frac{\lambda^2}{2\left( n_e - n_o \right) l}
} \\

Пользуясь формулой (\ref{WoodsFilterDelta}) можно рассчитать толщину фильтра, необходимого
для подавления одной из двух близких линий, разделенных интервалом $ \Delta\lambda $.

\newpage

\subsection{Температурная зависимость длины волны максимума пропускания}

Формула (\ref{WoodsFilterTransmissionMax}) позволяет определить температурный сдвиг
максимума пропускания фильтра. Для этого необходимо найти полную производную по $ T $
от этого выражения, учитывая, что и толщина пластинки $ l $, и коэф-т двойного
лучепреломления $ \mu = n_e - n_o $ зависят от температуры, а $ \mu $ есть так же
функция $ \lambda $. Дифференцируя, имеем:

\formula{TderivOfLambda}{
    \frac{d\lambda}{dT} =
    \frac{\frac{1}{\mu} \frac{\partial \mu}{\partial T} + \frac{1}{l} \frac{dl}{dT}}
         {\frac{1}{\lambda} - \frac{1}{\mu} \frac{\partial \mu}{\partial \lambda}}
} \\

Нетрудно видеть, что температурное смещение длинны волны не зависит от толщины пластинки
и определяется только свойствами материала, из которого она сделана (
$ \frac{1}{l} \frac{dl}{dT} $ -- коэф-т линейного расширения).

\subsection{Теромооптически компенсированный фильр Вуда}

Из эксперементальных данных следует, что для всех, используемых в подобных фильтрах
материалов, при повышении температуры $ \lambda_{max} $ смещается в синюю сторону.
Данное обстоятельство усложняет термооптическую компенсацию данного фильтра.
Однако, чтобы компенсировать температурные смещения, достаточно составить пластинку из
слоев двух материалов, ориентированных "на вычитание", т.е. так, чтобы обыкновенный
луч в первом слое становился необыкновенным во втором. Для этого нужно, чтоыб оптические
оси слоев были перпендикулярны -- см. Рисунок 3.

\pic{0.5\linewidth}{compensatedWoodsFilter.jpg}{
    Теромооптически компенсированный интерференционно-поляризационный фильр Вуда
}

Тогда условие максимума пропускания будет иметь вид:

\formula{compensatedWoodsFilterMax}{
    k\lambda = l_1\mu_1 - l_2\mu_2
} \\

Тогда, дифференцируя данное выражение получаем условие, про котором
$ \frac{d\lambda}{dT} = 0 $:

\formula{lambdaConstCondition}{
    \frac{l_1}{l_2} =
    \frac{\frac{\partial \mu_2}{\partial T} + \frac{\mu_2}{l_2} \frac{dl_2}{dT}}
         {\frac{\partial \mu_1}{\partial T} + \frac{\mu_1}{l_1} \frac{dl_1}{dT}}
} \\

Полностью компенсируя тепловые сдвиги, тонкая пластинка ADP (Двуводородный фосфат аммония)
лишь незначительно (на 6\%) уменьшает волновую разность хода.

\subsection{Апертура фильтра и требуемая точность обработки поверхностей}

Обычно можно считать приемлемыми смещения $ \delta\lambda $, меньшие, чем $ \frac{1}{5} $
полуширина полосы пропускания фильтра $ \Delta\lambda $. Тогда:

\formula{machiningAccuracy}{
    \delta l \leq  \frac{l}{\lambda} \frac{\Delta\lambda}{5} =
    \frac{\lambda}{10\mu}
} \\

Требуемая тончость обработки оказывается, таким образом, независимой от толщины пластинки.
Для кальцита ($ \mu = 0,17 $) она равна $ 0,3 $ мкм. Для кварца ($ \mu = 0,01 $)
$ \approx 6 $ мкм (для средней части спектра). \\

Для наклонных лучей света изменяется не только толщина слоя, но и его двойное
лучепреломление, причем по-разному, в зависимости от ориентации плоскости, в которой
лежит наклонный луч света. Кривые равной разности хода (изохроматы) предствляют собой
гиперболы -- см. Рисунок 4. Наибольший допустимый раствор светового пучка должен
соответствовать центральной части этой картины. Обычно он составляет угол, не
превышающий $ 1^\circ $.

\pic{0.5\linewidth}{izochroms.jpg}{Изохромы}
