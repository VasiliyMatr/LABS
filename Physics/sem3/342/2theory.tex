
% Theory part goes here %

% for numerated formulas
\newcommand{\formula}[3]
{
    \noindent#1\\[0.1cm]
    \begin{equation}\label{#2}
        #3
    \end{equation}
}

% for in-text math formulas
\newcommand{\mth}[1]
{
    \begin{math}
        #1
    \end{math}
}

% for rus letters in indexes
\newcommand{\ruB}[1]
{
    _{\text{#1}}
}

\section{Теория}

\formula
{При повышении температуры $T$ возрастает дезориентирующее действие теплового движения частиц, и магнитная восприимчивость ферромагнетиков убывает по закону Кюри-Вейсса:}
{CurieWeissLaw}
{\chi \propto \frac{1}{T - \Theta_p},}

где $\Theta_p$ - парамагнитная точка Кюри исследуемого вещества. При $T < \Theta_p$ образец обладает ферромагнитными свойствами и может сохранять намагниченность, при $T > \Theta_p$ образец ведёт себя как парамагнетик, для которого связь $B$ и $H$ однозначная: $I = \chi H$, $B = \mu H$. Для исследования выбран гадолиний, так как его точка Кюри лежит в интервале комнатных температур.

\section{Схема установки}

\pick{0.8\textwidth}{picks/scheme.png}{Схема установки}

Схема установки изображена на рис. 1. Исследуемый ферромагнитный образец (гадолиний) расположен внутри пустотелой катушки самоиндукции, которая служит индуктивностью колебательного контура, входящего в состав LC-автогенератора. Катушка с образцом помещена в стеклянный сосуд, залитый трансформаторным маслом. Температура образца регулируется с помощью термостата.
\formula
{При изменении температуры по закону Кюри-Вейсса изменяется магнитная восприимчивость образца в катушке и, следовательно, изменяется самоиндуктивность этой катушки. При этом изменяется период колебаний автогенератора. Поэтому получаем, что:}
{CurieWeissLawConsequence}
{\frac{1}{\chi} \sim (T - \Theta_p) \sim \frac{1}{(\tau^2 - \tau_0^2)},}

где $\tau$ и $\tau_0$ - период колебаний в цепи с сердечником в катушке и без него соответственно. Измерения проводятся в интервале температур от 14 $^{\circ}$С до 40 $^{\circ}$С
