\documentclass[11pt,a4paper,oneside]{article} % dock basic params

% ---------------- PACKAGES ---------------- %

% Ru lang stuff
    \usepackage[utf8x]{inputenc}
    \usepackage[T2A]{fontenc}

% running titles
    \usepackage{fancybox}
    \usepackage{fancyhdr}

% for last page number
    \usepackage{lastpage}

% for colored tablets cells
    \usepackage{colortbl}

% for Ru text in formulas
    \usepackage[warn]{mathtext}

% for captions
    \usepackage[labelsep=period]{caption}

% for colored hyperrefs
    \usepackage{xcolor}
    \usepackage{hyperref}

% for pictures
    \usepackage{graphicx}

% for coll math
    \usepackage{amsmath}

% path to all pictures
    \graphicspath{picks/}

% for enumerates
    \usepackage[shortlabels]{enumitem}

% for custom running titles
    \usepackage{ifthen}
    \usepackage{pdfpages}
    \usepackage[strict]{changepage}

% for simple drawings
    \usepackage{tikz}
    \usetikzlibrary{calc}
    \usetikzlibrary{decorations.pathmorphing}

% for good-styled text in tablets
    \usepackage{array}
    \newcolumntype{P}[1]{>{\centering\arraybackslash}p{#1}}

% dock fields & sizes
    \usepackage[left=12mm, top=12mm, right=15mm, bottom=28mm, nohead, footskip=10mm]{geometry}

% for multirows in tables
    \usepackage{multirow}

% ---------------- SOME USEFUL MACROS ---------------- %

% for shifted mini pages
    \newcommand{\shiftedText}[3]{
    \hspace*{#1}\begin{minipage}[t]{#2}
        #3
    \end{minipage}
    }

% lab num
    \newcommand{\labnum}{
        121
    }

% lab name
    \newcommand{\labname}{
        Измерение магнитного поля земли.
    }

% lab objective
    \newcommand{\labobj}{
        Определить характеристики шарообразных неодимовых магнитов и, используя законы взаимодействия магнитных моментов с полем, измерить горизонтальную и вертикальную составляющие индукции магнитного поля Земли и магнитное наклонение.
    }

% lab equipment
    \newcommand{\labequ}{
        12 одинаковых неодимовых магнитных шариков, тонкая нить для изготовления крутильного маятника, медная проволока диаметром (0,5 – 0,6) мм, электронные весы, секундомер, измеритель магнитной индукции АТЕ-8702, штангенциркуль, брусок из немагнитного материала (25x30x60 мм3), деревянная линейка, штатив из немагнитного материала; дополнительные неодимовые магнитные шарики (20 шт.) и неодимовые магниты в форме параллелепипедов (2 шт.), набор гирь и разновесов.
    }

% ---------------- PAGES STYLE & FORMATTING ---------------- %

% page style setup (for running titles)
    \fancypagestyle{plain}{
        \fancyhf{} % remove everything

         % lines parameters
        \renewcommand{\headrulewidth}{0pt}
        \renewcommand{\footrulewidth}{0pt}

        % running titles contents
        \fancyfoot[L]{\ifthenelse{\isodd{\thepage}}{Работа \labnum}{\thepage}}
        \fancyfoot[R]{\ifthenelse{\isodd{\thepage}}{\thepage}{Работа \labnum}}
    }

% choosing page style with our running titles
    \pagestyle{plain}

% to have line breaks without sizes excess
    \tolerance = 10000

% ---------------- DOC BODY ---------------- %

\begin{document}
    % some text placement parameters
        \textheight = 240mm
        \footskip = 10mm
        \leftskip = 10mm

    % add title page
        
% Custom title stuff here %

\shiftedText{0.5cm}{14cm}
{

    \begin{center}
    \vspace*{1.0cm}

        {\bf\huge Работа \labnum }

    \vspace*{0.2cm}

        {\bf\Large \labname }

    \vspace*{0.8cm}

        {\Large Работу выполнил Матренин Василий Б01-008 }

    \vspace*{1.6cm}

    \end{center}

    {\bf\noindent Цель работы: } \labobj

}

\newpage

    % fixing running titles shifts after title page
        \headheight = 0.5cm
        \headsep = 1.2cm

    % other files includes
        
% Theory part goes here %

% for numerated formulas
\newcommand{\formula}[2]
{
    \begin{equation}\label{#1}
        #2
    \end{equation}
}

% for in-text math formulas
\newcommand{\mth}[1]
{
    \begin{math}
        #1
    \end{math}
}

% for rus letters in indexes
\newcommand{\ruB}[1]
{
    _{\text{#1}}
}

\newcounter{PicsCounter}
\setcounter{PicsCounter}{1}

\newcommand{\pic}[3]{
    \begin{center}
        \begin{minipage}[h!]{#1}
            \begin{center}

                \includegraphics[width = \textwidth]{#2}
                \textit{Рис \arabic{PicsCounter}. #3}

            \end{center}
        \end{minipage}
    \end{center}

    \stepcounter{PicsCounter}
}

\newcounter{TablesCounter}
\setcounter{TablesCounter}{1}

\newcommand{\tableLable}[1]{
    \textit{Таблица \arabic{TablesCounter}: #1}

    \stepcounter{TablesCounter}
}

\section{Теория}

Основной задачей спектрометрических измерений является определение энергии и
интенсивности дискретных гамма-линий от различных гамма-источников и их идентификация.

\subsection{Процессы взаимодействия гамма-излучения с веществом}

\begin{enumerate}

    \item \textbf{Фотоэффект}

    Электрону передается вся энергия гамма-кванта. При этом электрону сообщается
    кинетическая энергия:
    \formula{PhotoelectricEffect}
    {T_e = E_\gamma - I_i \,}

    Где $ E_\gamma $ - энергия гамма-кванта, $ I_i $ - потенциал ионизации $ i-$той
    оболочки атома.

    \item \textbf{Эффект Комптона}

    Происходит упрогое рассеяниефотона на свободном электроне. Максимальная энергия
    образующихся комптоновских электронов соответствует рассеянию гамма-квантов на
    $ 180^\circ $ и равна:
    \formula{ComptonEffect}
    {E_{max} = \frac{\hslash\omega}{1 + \frac{mc^2}{2\hslash\omega}}}

    \item \textbf{Процесс образования электрон-позитронных пар}

    При достаточно высокой энергии гамма-квантов наряду с фотоэффектом и эффектом Комптона
    будет происходить образование электрон-позитронных пар.

    Вся энергия электрона останется в детекторе. Вся кинетическая энергия позитрона так же
    останется в детекторе. Затем позитрон аннигилирует с электроном среды. В результате чего
    образуются 2 гамма-кванта. Далее возможны 3 варианта развития событий:

    \begin{enumerate}

        \item Оба гамма-кванта остаются в детекторе. В спектре появляется пик
        $ E = E_\gamma $.

        \item Один гамма-квант покидает детектор. В спектре появляется пик, соответствующий
        энергии $ E = E_\gamma - E_0 \, $, где $ E_0 = mc^2 = 511 \, $ кЭв

        \item Оба гамма-кванта покидают детектор. В спектре появляется пик, соответствующий
        энергии $ E = E_\gamma - 2E_0 \, $, где $ E_0 = mc^2 = 511 \, $ кЭв

    \end{enumerate}

\end{enumerate}

Помимо этих процессов, добавляются экспонента, связанная с наличием фона, пик
характеристического излучения (возникающий при взаимодействии гамма-квантов с окружающим
веществом). А также пик обратного рассеяния, образующийся при энергии квантов
$ E_\gamma \gg mc^2 $ в результате рассеяния гамма-квантов на большие углы на материалах
конструктивных элементов детектора и защиты и последующего фотоэффекта в сцинтилляторе.

Положение пика обратного рассеяния определяется по формуле:

\formula{BackscatterMax}
{E\ruB{обр} = \frac{E}{1 + \frac{2E}{mc^2}}}

Где $ E $ - энергия фотопика.

\subsection{Энергетическое разрешение спектрометра}

Даже при поглощении частиц с одинаковой энергией, амплитуда импульса на выходе фотоприемника
сцинтилляционного детектора меняется от события к событию.

Энергетическим разрешением спектрометра называется величина:
\formula{SpectrometerResolution}
{R_i = \frac{\Delta E_i}{E_i}}

Где $ \Delta E_i $ - ширина пика полного поглощения, измеренная на половине высоты.

Из распределения Пуассона получаем:

\formula{SpectrometerResolutionFinal}
{R_i = \frac{const}{\sqrt{E_i}}}

\subsection{Схема установки}
Схема установки представлена на рисунке 1.

\pic{0.8\linewidth}{scheme1.jpg}{Схема установки}

        
% Experement data + calcs goes here %


\end{document}
